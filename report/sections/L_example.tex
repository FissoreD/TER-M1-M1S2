\subsection{L* with Exemple}
Let $U = regex(a + ba^*)$
\begin{enumerate}
  \item \quad
        \begin{center}
          \begin{tabular}{c || c }
            $T_1$         & $\varepsilon$ \\ [0.5ex]
            \hline\hline
            $\varepsilon$ & 1             \\
            \hline \hline
            a             & 1             \\
            b             & 0             \\
          \end{tabular}
          \quad
          \begin{minipage}{0.5\textwidth}
            At start we have $E = \{\varepsilon\}, S = \{\varepsilon\}, SA = \{a, b\}$,
            The table is not closed \\
            $row(b) = 0 \text{ but }$ \\
            $\nexists s \in S \text{ such that } row(s) = 0$ \\
            Row b is promoted and SA in updated
          \end{minipage}
        \end{center}
  \item \quad
        \begin{center}
          \begin{tabular}{c || c }
            $T_2$         & $\varepsilon$ \\ [0.5ex]
            \hline\hline
            $\varepsilon$ & 1             \\
            $b$           & 0             \\
            \hline \hline
            a             & 1             \\
            ba            & 1             \\
            bb            & 0             \\
          \end{tabular}
          \quad
          \begin{minipage}{0.5\textwidth}
            Row b is now in $S$ and $ba$ and $bb$ have been added in $SA$ to complete the table.\\
            Now, the table is closed and consistent an automaton will be sent.
          \end{minipage}
        \end{center}
  \item \quad
        \begin{center}
          \begin{tikzpicture}[>=latex',line join=bevel,scale=0.5]
  \pgfsetlinewidth{1bp}
  %%
  \pgfsetcolor{black}
  % Edge: 1 -> 1
  \draw [->] (52.683bp,42.991bp) .. controls (51.78bp,53.087bp) and (54.219bp,62.0bp)  .. (60.0bp,62.0bp) .. controls (63.704bp,62.0bp) and (66.036bp,58.342bp)  .. (67.317bp,42.991bp);
  \definecolor{strokecol}{rgb}{0.0,0.0,0.0};
  \pgfsetstrokecolor{strokecol}
  \draw (60.0bp,69.5bp) node {a};
  % Edge: 1 -> 0
  \draw [->] (82.117bp,22.0bp) .. controls (92.296bp,22.0bp) and (104.63bp,22.0bp)  .. (125.82bp,22.0bp);
  \draw (104.0bp,29.5bp) node {b};
  % Edge: 0 -> 1
  \draw [->] (128.69bp,12.134bp) .. controls (122.6bp,8.5065bp) and (115.24bp,4.8378bp)  .. (108.0bp,3.0bp) .. controls (101.24bp,1.2827bp) and (94.155bp,2.4513bp)  .. (78.473bp,9.1607bp);
  \draw (104.0bp,10.5bp) node {a};
  % Edge: 0 -> 0
  \draw [->] (136.97bp,38.664bp) .. controls (135.41bp,48.625bp) and (137.75bp,58.0bp)  .. (144.0bp,58.0bp) .. controls (148.0bp,58.0bp) and (150.4bp,54.152bp)  .. (151.03bp,38.664bp);
  \draw (144.0bp,65.5bp) node {b};
  % Edge: I1 -> 1
  \draw [->] (1.1664bp,22.0bp) .. controls (2.7183bp,22.0bp) and (14.889bp,22.0bp)  .. (37.856bp,22.0bp);
  % Node: 1
  \begin{scope}
    \definecolor{strokecol}{rgb}{0.0,0.0,0.0};
    \pgfsetstrokecolor{strokecol}
    \draw (60.0bp,22.0bp) ellipse (18.0bp and 18.0bp);
    \draw (60.0bp,22.0bp) ellipse (22.0bp and 22.0bp);
    \draw (60.0bp,22.0bp) node {1};
  \end{scope}
  % Node: 0
  \begin{scope}
    \definecolor{strokecol}{rgb}{0.0,0.0,0.0};
    \pgfsetstrokecolor{strokecol}
    \draw (144.0bp,22.0bp) ellipse (18.0bp and 18.0bp);
    \draw (144.0bp,22.0bp) node {0};
  \end{scope}
  %
\end{tikzpicture}
% End of code
          \quad
          \begin{minipage}{0.5\textwidth}
            This is the automaton sent. It has two states, one for the rows equal to 0 and one for
            the rows equal to 1. From state $1 = row(\varepsilon)$ if we read an $a$ we go rest in 1 since $row(\varepsilon \cdot a) = 1$. If we read a $b$ we go to 0 since $row(\varepsilon \cdot b) = 0$. 1 is the initial state since it is $row(\varepsilon) = 1$ and it is also an accepting state since the first bit of its row is a 1.\\
            A counter-exemple of this automaton after an equivalence query could be $aa$ since $aa$ is accepted by the A but not by the language of Teacher.
          \end{minipage}
        \end{center}
  \item  \quad
        \begin{center}
          \begin{tabular}{c || c }
            $T_3$         & $\varepsilon$ \\ [0.5ex]
            \hline\hline
            $\varepsilon$ & 1             \\
            $b$           & 0             \\
            $aa$          & 0             \\
            $a$           & 1             \\
            \hline \hline
            ba            & 1             \\
            bb            & 0             \\
            aaa           & 0             \\
            aab           & 0             \\
            ab            & 0             \\
          \end{tabular}
          \quad
          \begin{minipage}{0.5\textwidth}
            The table is closed but not consistent :\\
            $row(\varepsilon) = row(a)$ but $row(\varepsilon \cdot a) \neq row(a \cdot a)$.\\
            We can distinguish $row(\varepsilon)$ and $row(a)$ by adding $a$ to $E$ since $T(a \cdot a \cdot \varepsilon) \neq T(\varepsilon \cdot a \cdot \varepsilon)$. \\
            Note : in $a \cdot a \cdot \varepsilon$, the first $a$ is taken from $S$, the second from $\Sigma$ and $\varepsilon$ from $E$.
          \end{minipage}
        \end{center}
  \item  \quad
        \begin{center}
          \begin{tabular}{c || c | c }
            $T_4$         & $\varepsilon$ & a \\ [0.5ex]
            \hline\hline
            $\varepsilon$ & 1             & 1 \\
            $b$           & 0             & 1 \\
            $aa$          & 0             & 0 \\
            $a$           & 1             & 0 \\
            \hline \hline
            ba            & 1             & 0 \\
            bb            & 0             & 0 \\
            aaa           & 0             & 0 \\
            aab           & 0             & 0 \\
            ab            & 0             & 0 \\
          \end{tabular}
          \quad
          \begin{minipage}{0.5\textwidth}
            The table is now closed and consistent.
            It is easy to check the closedness (every line in the lower part are also in the upper part), and the consistence (there are no two similar rows in the upper part).
          \end{minipage}
        \end{center}
  \item \quad
        \begin{center}
          % Start of code
% \begin{tikzpicture}[anchor=mid,>=latex',line join=bevel,]
  \begin{tikzpicture}[>=latex',line join=bevel,scale=0.5]
    \pgfsetlinewidth{1bp}
  %%
  \pgfsetcolor{black}
    % Edge: 10 -> 01
    \draw [->] (83.509bp,48.397bp) .. controls (94.723bp,53.147bp) and (108.65bp,59.044bp)  .. (130.16bp,68.155bp);
    \definecolor{strokecol}{rgb}{0.0,0.0,0.0};
    \pgfsetstrokecolor{strokecol}
    \draw (107.0bp,67.998bp) node {b};
    % Edge: 10 -> 11
    \draw [->] (84.957bp,37.022bp) .. controls (97.84bp,35.63bp) and (114.32bp,33.897bp)  .. (129.0bp,32.498bp) .. controls (153.31bp,30.182bp) and (180.79bp,27.846bp)  .. (211.84bp,25.3bp);
    \draw (148.49bp,39.998bp) node {a};
    % Edge: 01 -> 11
    \draw [->] (165.48bp,65.717bp) .. controls (177.0bp,58.67bp) and (192.85bp,48.977bp)  .. (214.99bp,35.426bp);
    \draw (189.99bp,60.998bp) node {b};
    % Edge: 01 -> 00
    \draw [->] (166.29bp,83.796bp) .. controls (178.39bp,89.779bp) and (194.96bp,97.964bp)  .. (217.73bp,109.22bp);
    \draw (189.99bp,104.0bp) node {a};
    % Edge: 11 -> 11
    \draw [->] (227.4bp,45.699bp) .. controls (226.52bp,55.998bp) and (229.22bp,64.997bp)  .. (235.49bp,64.997bp) .. controls (239.51bp,64.997bp) and (242.06bp,61.304bp)  .. (243.58bp,45.699bp);
    \draw (235.49bp,72.497bp) node {a,b};
    % Edge: 00 -> 00
    \draw [->] (227.64bp,135.63bp) .. controls (226.17bp,145.71bp) and (228.79bp,155.0bp)  .. (235.49bp,155.0bp) .. controls (239.79bp,155.0bp) and (242.4bp,151.19bp)  .. (243.35bp,135.63bp);
    \draw (235.49bp,162.5bp) node {a,b};
    % Edge: I10 -> 10
    \draw [->] (1.1707bp,39.498bp) .. controls (2.7304bp,39.498bp) and (14.746bp,39.498bp)  .. (37.824bp,39.498bp);
    % Node: 10
  \begin{scope}
    \definecolor{strokecol}{rgb}{0.0,0.0,0.0};
    \pgfsetstrokecolor{strokecol}
    \draw (61.5bp,39.5bp) ellipse (19.5bp and 19.5bp);
    \draw (61.5bp,39.5bp) ellipse (23.5bp and 23.5bp);
    \draw (61.498bp,39.498bp) node {10};
  \end{scope}
    % Node: 01
  \begin{scope}
    \definecolor{strokecol}{rgb}{0.0,0.0,0.0};
    \pgfsetstrokecolor{strokecol}
    \draw (148.49bp,75.5bp) ellipse (19.5bp and 19.5bp);
    \draw (148.49bp,75.498bp) node {01};
  \end{scope}
    % Node: 11
  \begin{scope}
    \definecolor{strokecol}{rgb}{0.0,0.0,0.0};
    \pgfsetstrokecolor{strokecol}
    \draw (235.49bp,23.5bp) ellipse (19.5bp and 19.5bp);
    \draw (235.49bp,23.5bp) ellipse (23.5bp and 23.5bp);
    \draw (235.49bp,23.498bp) node {11};
  \end{scope}
    % Node: 00
  \begin{scope}
    \definecolor{strokecol}{rgb}{0.0,0.0,0.0};
    \pgfsetstrokecolor{strokecol}
    \draw (235.49bp,117.5bp) ellipse (19.5bp and 19.5bp);
    \draw (235.49bp,117.5bp) node {00};
  \end{scope}
  %
  \end{tikzpicture}
  % End of code
  
          \quad
          \begin{minipage}{0.5\textwidth}
            The automaton sent is still not valid : \\
            a counter-exemple could be $baba$ since $baba$ is accepted \\
            by the Teacher but not by the automaton.\\
            The counter-example and all of its prefixes will be added to $S$.
          \end{minipage}
        \end{center}
  \item \quad
        \begin{center}
          \begin{tabular}{c || c | c }
            $T_5$         & $\varepsilon$ & a \\ [0.5ex]
            \hline\hline
            $\varepsilon$ & 1             & 1 \\
            $b$           & 0             & 1 \\
            $aa$          & 0             & 0 \\
            $a$           & 1             & 0 \\
            $baba$        & 1             & 0 \\
            $bab$         & 0             & 1 \\
            $ba$          & 1             & 0 \\
            \hline \hline
            ba            & 1             & 0 \\
            bb            & 0             & 0 \\
            aaa           & 0             & 0 \\
            aab           & 0             & 0 \\
            ab            & 0             & 0 \\
            $babaa$       & 0             & 0 \\
            $babab$       & 0             & 1 \\
            $babb$        & 0             & 0 \\
            $baa$         & 0             & 0 \\
          \end{tabular}
          \quad
          \begin{minipage}{0.5\textwidth}
            The table is closed but not consistent :\\
            $row(a) = row(baba)$ but $row(a \cdot b) \neq row(baba \cdot b)$.\\
            We can distinguish them by adding $ba$ to $E$ since $T(a \cdot b \cdot a) \neq T(baba \cdot b \cdot a)$. \\
            Note : in $a \cdot b \cdot a$, the first $a$ is taken from $S$, the $b$ from $\Sigma$ and the second $a$ from $E$.
          \end{minipage}
        \end{center}
  \item \quad
        \begin{center}
          \begin{tabular}{c || c | c | c }
            $T_6$         & $\varepsilon$ & a & ba \\ [0.5ex]
            \hline\hline
            $\varepsilon$ & 1             & 1 & 1  \\
            $b$           & 0             & 1 & 0  \\
            $aa$          & 0             & 0 & 0  \\
            $a$           & 1             & 0 & 0  \\
            $baba$        & 1             & 0 & 1  \\
            $bab$         & 0             & 1 & 0  \\
            $ba$          & 1             & 0 & 1  \\
            \hline \hline
            ba            & 1             & 0 & 0  \\
            bb            & 0             & 0 & 0  \\
            aaa           & 0             & 0 & 0  \\
            aab           & 0             & 0 & 0  \\
            ab            & 0             & 0 & 0  \\
            $babaa$       & 0             & 0 & 0  \\
            $babav$       & 0             & 1 & 0  \\
            $babb$        & 0             & 0 & 0  \\
            $baa$         & 0             & 0 & 0  \\
          \end{tabular}
          \quad
          \begin{minipage}{0.5\textwidth}
            The table is closed and consistent.\\
            An automaton will be sent.
          \end{minipage}
        \end{center}
  \item \quad
        \begin{center}
          % Start of code
% \begin{tikzpicture}[anchor=mid,>=latex',line join=bevel,]
\begin{tikzpicture}[>=latex',line join=bevel,scale=0.5]
  \pgfsetlinewidth{1bp}
  %%
  \pgfsetcolor{black}
  % Edge: 111 -> 100
  \draw [->] (93.625bp,86.749bp) .. controls (122.46bp,93.36bp) and (168.67bp,103.96bp)  .. (210.75bp,113.6bp);
  \definecolor{strokecol}{rgb}{0.0,0.0,0.0};
  \pgfsetstrokecolor{strokecol}
  \draw (116.1bp,100.15bp) node {a};
  % Edge: 111 -> 010
  \draw [->] (93.055bp,72.884bp) .. controls (104.46bp,69.476bp) and (118.0bp,65.43bp)  .. (140.02bp,58.85bp);
  \draw (116.1bp,75.155bp) node {b};
  % Edge: 100 -> 000
  \draw [->] (265.89bp,114.94bp) .. controls (286.61bp,110.84bp) and (315.81bp,104.07bp)  .. (340.29bp,94.655bp) .. controls (354.24bp,89.288bp) and (368.94bp,81.704bp)  .. (389.98bp,69.698bp);
  \draw (312.24bp,117.15bp) node {a,b};
  % Edge: 010 -> 101
  \draw [->] (192.38bp,52.742bp) .. controls (215.21bp,53.685bp) and (248.54bp,55.063bp)  .. (284.17bp,56.536bp);
  \draw (238.14bp,64.155bp) node {a};
  % Edge: 010 -> 000
  \draw [->] (185.15bp,39.386bp) .. controls (217.44bp,20.195bp) and (284.82bp,-12.906bp)  .. (340.29bp,5.655bp) .. controls (358.02bp,11.59bp) and (374.85bp,24.079bp)  .. (395.08bp,42.464bp);
  \draw (312.24bp,13.155bp) node {b};
  % Edge: 101 -> 010
  \draw [->] (287.74bp,43.034bp) .. controls (281.03bp,39.596bp) and (273.53bp,36.421bp)  .. (266.19bp,34.655bp) .. controls (241.95bp,28.825bp) and (234.6bp,30.085bp)  .. (210.1bp,34.655bp) .. controls (205.84bp,35.449bp) and (201.46bp,36.64bp)  .. (187.56bp,41.543bp);
  \draw (238.14bp,42.155bp) node {b};
  % Edge: 101 -> 000
  \draw [->] (340.51bp,57.655bp) .. controls (350.68bp,57.655bp) and (362.39bp,57.655bp)  .. (383.28bp,57.655bp);
  \draw (361.79bp,65.155bp) node {a};
  % Edge: 000 -> 000
  \draw [->] (401.72bp,75.065bp) .. controls (400.12bp,84.743bp) and (402.97bp,93.655bp)  .. (410.29bp,93.655bp) .. controls (414.86bp,93.655bp) and (417.69bp,90.174bp)  .. (418.86bp,75.065bp);
  \draw (410.29bp,101.15bp) node {a,b};
  % Edge: I111 -> 111
  \draw [->] (1.0464bp,80.655bp) .. controls (1.8856bp,80.655bp) and (13.98bp,80.655bp)  .. (37.822bp,80.655bp);
  % Node: 111
  \begin{scope}
    \definecolor{strokecol}{rgb}{0.0,0.0,0.0};
    \pgfsetstrokecolor{strokecol}
    \draw (66.05bp,80.65bp) ellipse (24.09bp and 24.09bp);
    \draw (66.05bp,80.65bp) ellipse (28.1bp and 28.1bp);
    \draw (66.048bp,80.655bp) node {111};
  \end{scope}
  % Node: 100
  \begin{scope}
    \definecolor{strokecol}{rgb}{0.0,0.0,0.0};
    \pgfsetstrokecolor{strokecol}
    \draw (238.14bp,119.65bp) ellipse (24.09bp and 24.09bp);
    \draw (238.14bp,119.65bp) ellipse (28.1bp and 28.1bp);
    \draw (238.14bp,119.65bp) node {100};
  \end{scope}
  % Node: 010
  \begin{scope}
    \definecolor{strokecol}{rgb}{0.0,0.0,0.0};
    \pgfsetstrokecolor{strokecol}
    \draw (165.1bp,51.65bp) ellipse (27.0bp and 18.0bp);
    \draw (165.1bp,51.655bp) node {010};
  \end{scope}
  % Node: 000
  \begin{scope}
    \definecolor{strokecol}{rgb}{0.0,0.0,0.0};
    \pgfsetstrokecolor{strokecol}
    \draw (410.29bp,57.65bp) ellipse (27.0bp and 18.0bp);
    \draw (410.29bp,57.655bp) node {000};
  \end{scope}
  % Node: 101
  \begin{scope}
    \definecolor{strokecol}{rgb}{0.0,0.0,0.0};
    \pgfsetstrokecolor{strokecol}
    \draw (312.24bp,57.65bp) ellipse (24.09bp and 24.09bp);
    \draw (312.24bp,57.65bp) ellipse (28.1bp and 28.1bp);
    \draw (312.24bp,57.655bp) node {101};
  \end{scope}
  %
\end{tikzpicture}
% End of code
          \quad
          \begin{minipage}{0.5\textwidth}
            $L(A) = U \rightarrow$ the Teacher accepts the automaton.
          \end{minipage}
        \end{center}
\end{enumerate}