\section{L* and NL* execution with example}

This section aims to provide an example of the execution of the two algorithms taking the unknown language $\U$ defined by $\E+(b+aa+abb)(a+b)^*$ % $+(b+aa+abb)(a+b)*
\subsection{L* Execution}
\label{sec:l-exec}
\begin{enumerate}
  \item \begin{minipage}{0.3\textwidth}
          \begin{tabular}{c||c}
            $T_1$ & $\E$ \\
            \hline\hline
            $\E$  & 1    \\
            \hline\hline
            b     & 1    \\
            a     & 0    \\
          \end{tabular}
        \end{minipage} \quad
        \begin{minipage}{0.6\textwidth}
          The table is not closed, because $row(a) = 0$ is not present in S. \\
          $b$ will be promoted.
        \end{minipage}
  \item \begin{minipage}{0.3\textwidth}
          \begin{tabular}{c||c}
            $T_2$ & $\E$ \\
            \hline\hline
            $\E$  & 1    \\
            a     & 0    \\
            \hline\hline
            b     & 1    \\
            ab    & 0    \\
            aa    & 1    \\
          \end{tabular}
        \end{minipage}\quad
        \begin{minipage}{0.6\textwidth}
          $ab$ and $aa$ have been added in $SA$ to respect the condition where: $\forall x \in S, \forall y \in \Sigma, \exists z \in S_{ext} \text{ such that } x \cdot y = z $.\\
          The table is closed ($\forall x \in SA, \exists s \in S \mid row(x) = row(s)$) and consistent (there are no two similar rows in $S$), the \textit{Learner} can send its conjecture.
        \end{minipage}

  \item \begin{minipage}{0.3\textwidth}
          \begin{tikzpicture}[>=latex',line join=bevel,scale=0.5]
  \pgfsetlinewidth{1bp}
  %%
  \pgfsetcolor{black}
  % Edge: 1 -> 1
  \draw [->] (52.683bp,42.991bp) .. controls (51.78bp,53.087bp) and (54.219bp,62.0bp)  .. (60.0bp,62.0bp) .. controls (63.704bp,62.0bp) and (66.036bp,58.342bp)  .. (67.317bp,42.991bp);
  \definecolor{strokecol}{rgb}{0.0,0.0,0.0};
  \pgfsetstrokecolor{strokecol}
  \draw (60.0bp,69.5bp) node {a};
  % Edge: 1 -> 0
  \draw [->] (82.117bp,22.0bp) .. controls (92.296bp,22.0bp) and (104.63bp,22.0bp)  .. (125.82bp,22.0bp);
  \draw (104.0bp,29.5bp) node {b};
  % Edge: 0 -> 1
  \draw [->] (128.69bp,12.134bp) .. controls (122.6bp,8.5065bp) and (115.24bp,4.8378bp)  .. (108.0bp,3.0bp) .. controls (101.24bp,1.2827bp) and (94.155bp,2.4513bp)  .. (78.473bp,9.1607bp);
  \draw (104.0bp,10.5bp) node {a};
  % Edge: 0 -> 0
  \draw [->] (136.97bp,38.664bp) .. controls (135.41bp,48.625bp) and (137.75bp,58.0bp)  .. (144.0bp,58.0bp) .. controls (148.0bp,58.0bp) and (150.4bp,54.152bp)  .. (151.03bp,38.664bp);
  \draw (144.0bp,65.5bp) node {b};
  % Edge: I1 -> 1
  \draw [->] (1.1664bp,22.0bp) .. controls (2.7183bp,22.0bp) and (14.889bp,22.0bp)  .. (37.856bp,22.0bp);
  % Node: 1
  \begin{scope}
    \definecolor{strokecol}{rgb}{0.0,0.0,0.0};
    \pgfsetstrokecolor{strokecol}
    \draw (60.0bp,22.0bp) ellipse (18.0bp and 18.0bp);
    \draw (60.0bp,22.0bp) ellipse (22.0bp and 22.0bp);
    \draw (60.0bp,22.0bp) node {1};
  \end{scope}
  % Node: 0
  \begin{scope}
    \definecolor{strokecol}{rgb}{0.0,0.0,0.0};
    \pgfsetstrokecolor{strokecol}
    \draw (144.0bp,22.0bp) ellipse (18.0bp and 18.0bp);
    \draw (144.0bp,22.0bp) node {0};
  \end{scope}
  %
\end{tikzpicture}
% End of code
        \end{minipage}\quad
        \begin{minipage}{0.6\textwidth}
          This is the automaton sent.
          The states are $0$ and $1$ since there are two distinct rows in the upper part of the table, $1$ is the initial state since $row(\E) = 0$ and it is also accepting since $T(\E) = 1$.\\
          Transitions are made as follow: \\
          \begin{itemize}
            \item $\delta(1, a) = \delta(row(\E), a) = row(\E \cdot a) = 0$.
            \item $\delta(1, b) = \delta(row(\E), b) = row(\E \cdot b) =  1$.
            \item $\delta(0, a) = \delta(row(a), a) = row(a \cdot a) =  1$.
            \item $\delta(0, b) = \delta(row(a), b) = row(a \cdot b) = 0$.
          \end{itemize}
          However, the automaton is not valid, since the word \textit{aba} is accepted by the conjecture but not by the \textit{Teacher}.
        \end{minipage}

  \item \begin{minipage}{0.3\textwidth}
          \begin{tabular}{c||c}
            $T_3$ & $\E$ \\
            \hline\hline
            $\E$  & 1    \\
            a     & 0    \\
            aba   & 0    \\
            ab    & 0    \\
            \hline\hline
            b     & 1    \\
            aa    & 1    \\
            abab  & 0    \\
            abaa  & 0    \\
            abb   & 1    \\
          \end{tabular}
        \end{minipage} \quad
        \begin{minipage}{0.5\textwidth}
          $aba$ as been added in $S$ as well as each of its prefixes. $abab, abaa \text{ and } abb$ have been added in $SA$ to keep the \OT complete. \\
          This table is not consistent: $row(a) = row(aba)$ but taking $row(a \cdot a) \neq row(aba \cdot abaa)$. Column $a$ is going to be added because $T(a \cdot a \cdot \E) \neq row(aba \cdot a \cdot \E)$
        \end{minipage}

  \item \begin{minipage}{0.3\textwidth}
          \begin{tabular}{c||c|c}
            $T_4$ & $\E$ & a \\
            \hline\hline
            $\E$  & 1    & 0 \\
            a     & 0    & 1 \\
            aba   & 0    & 0 \\
            ab    & 0    & 0 \\
            \hline\hline
            b     & 1    & 1 \\
            aa    & 1    & 1 \\
            abab  & 0    & 0 \\
            abaa  & 0    & 0 \\
            abb   & 1    & 1 \\
          \end{tabular}
        \end{minipage} \quad
        \begin{minipage}{0.5\textwidth}
          $T_43$ is not closed: $row(b)$ is not present in $S$, so it will be promoted.
        \end{minipage}

  \item \begin{minipage}{0.3\textwidth}
          \begin{tabular}{c||c|c}
            $T_5$ & $\E$ & a \\
            \hline\hline
            $\E$  & 1    & 0 \\
            a     & 0    & 1 \\
            aba   & 0    & 0 \\
            ab    & 0    & 0 \\
            b     & 1    & 1 \\
            \hline\hline
            aa    & 1    & 1 \\
            abab  & 0    & 0 \\
            abaa  & 0    & 0 \\
            abb   & 1    & 1 \\
            bb    & 1    & 1 \\
            ba    & 1    & 1 \\
          \end{tabular}
        \end{minipage} \quad
        \begin{minipage}{0.5\textwidth}
          $T_5$ is not consistent $row(aba) = row(ab)$ but $row(abab) \neq row(abb)$
          Column $b$ is going to be added because $T(aba \cdot b \cdot \E) \neq row(ab \cdot b \cdot \E)$
        \end{minipage}

  \item \begin{minipage}{0.3\textwidth}
          \begin{tabular}{c||c|c|c}
            $T_6$ & $\E$ & a & b \\
            \hline\hline
            $\E$  & 1    & 0 & 1 \\
            a     & 0    & 1 & 0 \\
            aba   & 0    & 0 & 0 \\
            ab    & 0    & 0 & 1 \\
            b     & 1    & 1 & 1 \\
            \hline\hline
            aa    & 1    & 1 & 1 \\
            abab  & 0    & 0 & 0 \\
            abaa  & 0    & 0 & 0 \\
            abb   & 1    & 1 & 1 \\
            bb    & 1    & 1 & 1 \\
            ba    & 1    & 1 & 1 \\
          \end{tabular}
        \end{minipage} \quad
        \begin{minipage}{0.5\textwidth}
          $T_6$ is closed and consistent.
        \end{minipage}

  \item \begin{minipage}{0.3\textwidth}
          % Start of code
% \begin{tikzpicture}[anchor=mid,>=latex',line join=bevel,]
  \begin{tikzpicture}[>=latex',line join=bevel,scale=0.5]
    \pgfsetlinewidth{1bp}
  %%
  \pgfsetcolor{black}
    % Edge: 10 -> 01
    \draw [->] (83.509bp,48.397bp) .. controls (94.723bp,53.147bp) and (108.65bp,59.044bp)  .. (130.16bp,68.155bp);
    \definecolor{strokecol}{rgb}{0.0,0.0,0.0};
    \pgfsetstrokecolor{strokecol}
    \draw (107.0bp,67.998bp) node {b};
    % Edge: 10 -> 11
    \draw [->] (84.957bp,37.022bp) .. controls (97.84bp,35.63bp) and (114.32bp,33.897bp)  .. (129.0bp,32.498bp) .. controls (153.31bp,30.182bp) and (180.79bp,27.846bp)  .. (211.84bp,25.3bp);
    \draw (148.49bp,39.998bp) node {a};
    % Edge: 01 -> 11
    \draw [->] (165.48bp,65.717bp) .. controls (177.0bp,58.67bp) and (192.85bp,48.977bp)  .. (214.99bp,35.426bp);
    \draw (189.99bp,60.998bp) node {b};
    % Edge: 01 -> 00
    \draw [->] (166.29bp,83.796bp) .. controls (178.39bp,89.779bp) and (194.96bp,97.964bp)  .. (217.73bp,109.22bp);
    \draw (189.99bp,104.0bp) node {a};
    % Edge: 11 -> 11
    \draw [->] (227.4bp,45.699bp) .. controls (226.52bp,55.998bp) and (229.22bp,64.997bp)  .. (235.49bp,64.997bp) .. controls (239.51bp,64.997bp) and (242.06bp,61.304bp)  .. (243.58bp,45.699bp);
    \draw (235.49bp,72.497bp) node {a,b};
    % Edge: 00 -> 00
    \draw [->] (227.64bp,135.63bp) .. controls (226.17bp,145.71bp) and (228.79bp,155.0bp)  .. (235.49bp,155.0bp) .. controls (239.79bp,155.0bp) and (242.4bp,151.19bp)  .. (243.35bp,135.63bp);
    \draw (235.49bp,162.5bp) node {a,b};
    % Edge: I10 -> 10
    \draw [->] (1.1707bp,39.498bp) .. controls (2.7304bp,39.498bp) and (14.746bp,39.498bp)  .. (37.824bp,39.498bp);
    % Node: 10
  \begin{scope}
    \definecolor{strokecol}{rgb}{0.0,0.0,0.0};
    \pgfsetstrokecolor{strokecol}
    \draw (61.5bp,39.5bp) ellipse (19.5bp and 19.5bp);
    \draw (61.5bp,39.5bp) ellipse (23.5bp and 23.5bp);
    \draw (61.498bp,39.498bp) node {10};
  \end{scope}
    % Node: 01
  \begin{scope}
    \definecolor{strokecol}{rgb}{0.0,0.0,0.0};
    \pgfsetstrokecolor{strokecol}
    \draw (148.49bp,75.5bp) ellipse (19.5bp and 19.5bp);
    \draw (148.49bp,75.498bp) node {01};
  \end{scope}
    % Node: 11
  \begin{scope}
    \definecolor{strokecol}{rgb}{0.0,0.0,0.0};
    \pgfsetstrokecolor{strokecol}
    \draw (235.49bp,23.5bp) ellipse (19.5bp and 19.5bp);
    \draw (235.49bp,23.5bp) ellipse (23.5bp and 23.5bp);
    \draw (235.49bp,23.498bp) node {11};
  \end{scope}
    % Node: 00
  \begin{scope}
    \definecolor{strokecol}{rgb}{0.0,0.0,0.0};
    \pgfsetstrokecolor{strokecol}
    \draw (235.49bp,117.5bp) ellipse (19.5bp and 19.5bp);
    \draw (235.49bp,117.5bp) node {00};
  \end{scope}
  %
  \end{tikzpicture}
  % End of code
  
        \end{minipage}\quad\\
        \begin{minipage}{0.6\textwidth}
          This automaton recognized perfectly the language proposed by the Teacher, so the algorithm can stop.
        \end{minipage}
\end{enumerate}
\subsection{NL* Execution}
\label{sec:nl-exec}
\begin{enumerate}
  \item \begin{minipage}{0.3\textwidth}
          \begin{tabular}{c||c}
            $T_1$              & $\E$ \\
            \hline\hline
            *$\E$\footnotemark & 1    \\
            \hline\hline
            *b                 & 1    \\
            *a                 & 0    \\
          \end{tabular}
        \end{minipage} \footnotetext{Prime rows are indicated by a leading asterisk} \quad
        \begin{minipage}{0.6\textwidth}
          The table is not closed, because $row(a) = 0$ but there is no $s \in S$ such that $row(s) \subseteq 0$ . $a$ will be promoted.
        \end{minipage}

  \item \begin{minipage}{0.3\textwidth}
          \begin{tabular}{c||c}
            $T_2$ & $\E$ \\
            \hline\hline
            *$\E$ & 1    \\
            *a    & 0    \\
            \hline\hline
            *b    & 1    \\
            *ab   & 0    \\
            *aa   & 1    \\
          \end{tabular}
        \end{minipage}\quad
        \begin{minipage}{0.6\textwidth}
          $T_2$ is not consistent: $row(a) \sqsubseteq row(\E)$ but $row(a \cdot a) \not\sqsubseteq row(\E \cdot a)$. Column $a$ is going to be added because $T(a \cdot a \cdot \E) \neq row(\E \cdot a \cdot \E)$.
        \end{minipage}

  \item \begin{minipage}{0.3\textwidth}
          \begin{tabular}{c||c |c}
            $T_3$ & $\E$ & a \\
            \hline\hline
            *$\E$ & 1    & 0 \\
            *a    & 0    & 1 \\
            \hline\hline
            *b    & 1    & 1 \\
            *ab   & 0    & 0 \\
            *aa   & 1    & 1 \\
          \end{tabular}
        \end{minipage}\quad
        \begin{minipage}{0.6\textwidth}
          $T_3$ is not closed since $row(ab) = 00$ but $\nexists x \in S$ such that $x \sqsubseteq row(ab)$
        \end{minipage}

  \item \begin{minipage}{0.3\textwidth}
          \begin{tabular}{c||c |c}
            $T_4$ & $\E$ & a \\
            \hline\hline
            *$\E$ & 1    & 0 \\
            *a    & 0    & 1 \\
            *ab   & 0    & 0 \\
            \hline\hline
            b     & 1    & 1 \\
            aa    & 1    & 1 \\
            abb   & 1    & 1 \\
            *aba  & 0    & 0 \\
          \end{tabular}
        \end{minipage}\quad
        \begin{minipage}{0.6\textwidth}
          $T_4$ is not consistent: $row(ab) \sqsubseteq row(a)$ but $row(ab \cdot b) \not\sqsubseteq row(a \cdot b)$. Column $b$ is going to be added because $T(ab \cdot b \cdot \E) \neq row(a \cdot b \cdot \E)$.
        \end{minipage}

  \item \begin{minipage}{0.3\textwidth}
          \begin{tabular}{c||c |c|c}
            $T_5$ & $\E$ & a & b \\
            \hline\hline
            *$\E$ & 1    & 0 & 1 \\
            *a    & 0    & 1 & 0 \\
            *ab   & 0    & 0 & 1 \\
            \hline\hline
            b     & 1    & 1 & 1 \\
            aa    & 1    & 1 & 1 \\
            abb   & 1    & 1 & 1 \\
            *aba  & 0    & 0 & 0 \\
          \end{tabular}
        \end{minipage}\quad
        \begin{minipage}{0.6\textwidth}
          $T_5$ is not closed since $row(aba) = 000$ but $\nexists x \in S$ such that $x \sqsubseteq row(aba)$.\\
          $aba$ will be promoted.
        \end{minipage}

  \item \begin{minipage}{0.3\textwidth}
          \begin{tabular}{c||c |c|c}
            $T_6$ & $\E$ & a & b \\
            \hline\hline
            *$\E$ & 1    & 0 & 1 \\
            *a    & 0    & 1 & 0 \\
            *ab   & 0    & 0 & 1 \\
            *aba  & 0    & 0 & 0 \\
            \hline\hline
            b     & 1    & 1 & 1 \\
            aa    & 1    & 1 & 1 \\
            abb   & 1    & 1 & 1 \\
            *abab & 0    & 0 & 0 \\
            *abaa & 0    & 0 & 0 \\
          \end{tabular}
        \end{minipage}\quad
        \begin{minipage}{0.6\textwidth}
          $T_6$ is closed and consistent.\\
          The conjecture will be sent.
        \end{minipage}

  \item \begin{minipage}{0.3\textwidth}
          % Start of code
% \begin{tikzpicture}[anchor=mid,>=latex',line join=bevel,]
\begin{tikzpicture}[>=latex',line join=bevel,scale=0.5]
  \pgfsetlinewidth{1bp}
  %%
  \pgfsetcolor{black}
  % Edge: 1 -> 1
  \draw [->] (52.683bp,42.991bp) .. controls (51.78bp,53.087bp) and (54.219bp,62.0bp)  .. (60.0bp,62.0bp) .. controls (63.704bp,62.0bp) and (66.036bp,58.342bp)  .. (67.317bp,42.991bp);
  \definecolor{strokecol}{rgb}{0.0,0.0,0.0};
  \pgfsetstrokecolor{strokecol}
  \draw (60.0bp,69.5bp) node {a};
  % Edge: 1 -> 0
  \draw [->] (80.511bp,30.059bp) .. controls (94.225bp,35.723bp) and (112.7bp,43.356bp)  .. (137.18bp,53.468bp);
  \draw (109.0bp,52.5bp) node {a,b};
  % Edge: 0 -> 1
  \draw [->] (144.49bp,44.519bp) .. controls (138.44bp,35.239bp) and (129.39bp,24.252bp)  .. (118.0bp,19.0bp) .. controls (109.85bp,15.242bp) and (100.33bp,14.518bp)  .. (81.396bp,16.42bp);
  \draw (109.0bp,26.5bp) node {b};
  % Edge: 0 -> 0
  \draw [->] (146.33bp,76.29bp) .. controls (144.48bp,86.389bp) and (147.04bp,96.0bp)  .. (154.0bp,96.0bp) .. controls (158.46bp,96.0bp) and (161.11bp,92.056bp)  .. (161.67bp,76.29bp);
  \draw (154.0bp,103.5bp) node {a,b};
  % Edge: I1 -> 1
  \draw [->] (1.1664bp,22.0bp) .. controls (2.7183bp,22.0bp) and (14.889bp,22.0bp)  .. (37.856bp,22.0bp);
  % Edge: I0 -> 0
  \draw [->] (61.067bp,112.96bp) .. controls (62.907bp,111.9bp) and (102.41bp,89.143bp)  .. (137.83bp,68.741bp);
  % Node: 1
  \begin{scope}
    \definecolor{strokecol}{rgb}{0.0,0.0,0.0};
    \pgfsetstrokecolor{strokecol}
    \draw (60.0bp,22.0bp) ellipse (18.0bp and 18.0bp);
    \draw (60.0bp,22.0bp) ellipse (22.0bp and 22.0bp);
    \draw (60.0bp,22.0bp) node {1};
  \end{scope}
  % Node: 0
  \begin{scope}
    \definecolor{strokecol}{rgb}{0.0,0.0,0.0};
    \pgfsetstrokecolor{strokecol}
    \draw (154.0bp,60.0bp) ellipse (18.0bp and 18.0bp);
    \draw (154.0bp,60.0bp) node {0};
  \end{scope}
  %
\end{tikzpicture}
% End of code

        \end{minipage}\quad\\
        \begin{minipage}{1\textwidth}
          Construction of this automaton:
          \begin{enumerate}
            \item $Q = {101,010,001,000}$ which are the Prime rows in $S$;
            \item $Q_I = {101,001,000}$ which are all the rows that are covered by $row(\E)$;
            \item $F = {101}$ which is the only prime row having a $1$ in the column of $\E$;
            \item Transitions are more complicated to analyze, we will only describe those starting from the state $101$. We have $row(\E) = 101$ and we have to calculate the transition when reading:
                  \begin{enumerate}
                    \item $a \rightarrow row(\E \cdot a) = 010$ and the set of rows that are covered by $010$ is ${010, 000}$, so we draw two transitions labelled with $a$ from $101$ to $010$ and to $000$;
                    \item $b \rightarrow row(\E \cdot b) = 111$ and the set of rows that are covered by $111$ is ${010, 000, 101, 001}$, so we draw four transitions labelled with $a$ from $101$ to $010$, $000$, $101$ and $001$.
                  \end{enumerate}
          \end{enumerate}
          Moreover, this automaton recognizes precisely the language proposed by the Teacher, so the algorithm can stop.
        \end{minipage}
\end{enumerate}

\subsection{Example to find residual from a regular language}
In this section we would like to explain why the \textit{mDFA} of \cref{sec:l-exec} has one state more of the \textit{cRFSA} of \cref{sec:nl-exec}.

Let's start with the regular expression of $\U = \E+(b+aa+abb)(a+b)^*$. The residuals of this language can be found iteratively thanks \cref{algo:find_residuals}.


\begin{algorithm}[H]
  \caption{Algo to find residuals of a language $\U$}
  \label{algo:find_residuals}
  \KwIn{$\U$, $\Sigma$}
  \KwOut{$\RE$ = The set of residuals of $\U$}
  $\RE \gets \{\}$ \tcp*{Initialisation of $\RE$}
  $E \gets \{\E\}$\tcp*{The list of words to treat}
  \While{$E.length > 0$}{
    $current \gets E.pop()$\;
    $residual = (current)^{-1}L\footnote{This operation can be performed, for example, with the \textit{Brzozowski derivative} algorithm}$\;
    \If{$residual \notin \RE$}{
      $\RE \gets \RE \cup \{residual\}$\;
      \For{$sym \in \Sigma$}{
        $word \gets current \cdot sym$\;
        $E \gets E \cup \{word\}$\;
      }
    }
  }
  \Return $\RE$
\end{algorithm}

Let's apply this algorithm over $\U$.

\newcommand{\EM}{\varnothing}

\begin{tabular}{c|c|r|l|c|l}
  $Iteration$   & current & Operation      & Result                  & $\RE$                  & $E$                        \\
  \hline\hline
  \textit{Init} & $-$     & $-$            & $-$                     & $\RE = \{\}$           & $E = \{\E\}$               \\
  1             & $\E$    & $\E^{-1}\U$    & $\RE_1 = \U$            & $\RE = \RE \cup \RE_1$ & $E = \{a, b\}$             \\
  2             & $a$     & $a^{-1}\U$     & $\RE_2 = (a+bb)(a+b)^*$ & $\RE = \RE \cup \RE_2$ & $E = \{b, aa, ab\}$        \\
  3             & $b$     & $b^{-1}\U$     & $\RE_3 = (a+b)^*$       & $\RE = \RE \cup \RE_3$ & $E = \{aa, ab, ba, bb\}$   \\
  4             & $aa$    & $(aa)^{-1}\U$  & $(a+b)^* = \RE_3$       & $-$                    & $E = \{ab, ba, bb\}$       \\
  5             & $ab$    & $(ab)^{-1}\U$  & $\RE_4 = b(a+b)^*$      & $\RE = \RE \cup \RE_4$ & $E = \{ba, bb, aba, abb\}$ \\
  6             & $ba$    & $(ba)^{-1}\U$  & $(a+b)^* = \RE_3$       & $-$                    & $E = \{bb, aba, abb\}$     \\
  7             & $bb$    & $(bb)^{-1}\U$  & $(a+b)^* = \RE_3$       & $-$                    & $E = \{aba, abb\}$         \\
  8             & $aba$   & $(aba)^{-1}\U$ & $(a+b)^* = \RE_3$       & $-$                    & $E = \{abb\}$              \\
  9             & $abb$   & $(abb)^{-1}\U$ & $\RE_5 = \EM$           & $\RE = \RE \cup \RE_5$ & $E = \EM$                  \\
\end{tabular}

We see that the residuals of $\U$ are exactly five, the same number of state of the \textit{mDFA} returned by \textit{L*}, but only four of them are \textit{prime} because $\RE_3 = \RE_1 \cup \RE_2 \cup \RE_4$. That's why the \textit{cRFSA} has only four states.