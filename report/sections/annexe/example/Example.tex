\section{L* and NL* execution with example}

This section aims to provide an example of the execution of the two algorithms taking the unknown language $\U$ defined by $\E+(b+aa+abb)(a+b)^*$ % $+(b+aa+abb)(a+b)*
\subsection{L* with Exemple}
Let $U = regex(a + ba^*)$
\begin{enumerate}
  \item \quad
        \begin{center}
          \begin{tabular}{c || c }
            $T_1$         & $\varepsilon$ \\ [0.5ex]
            \hline\hline
            $\varepsilon$ & 1             \\
            \hline \hline
            a             & 1             \\
            b             & 0             \\
          \end{tabular}
          \quad
          \begin{minipage}{0.5\textwidth}
            At start we have $E = \{\varepsilon\}, S = \{\varepsilon\}, SA = \{a, b\}$,
            The table is not closed \\
            $row(b) = 0 \text{ but }$ \\
            $\nexists s \in S \text{ such that } row(s) = 0$ \\
            Row b is promoted and SA in updated
          \end{minipage}
        \end{center}
  \item \quad
        \begin{center}
          \begin{tabular}{c || c }
            $T_2$         & $\varepsilon$ \\ [0.5ex]
            \hline\hline
            $\varepsilon$ & 1             \\
            $b$           & 0             \\
            \hline \hline
            a             & 1             \\
            ba            & 1             \\
            bb            & 0             \\
          \end{tabular}
          \quad
          \begin{minipage}{0.5\textwidth}
            Row b is now in $S$ and $ba$ and $bb$ have been added in $SA$ to complete the table.\\
            Now, the table is closed and consistent an automaton will be sent.
          \end{minipage}
        \end{center}
  \item \quad
        \begin{center}
          \begin{tikzpicture}[>=latex',line join=bevel,scale=0.5]
  \pgfsetlinewidth{1bp}
  %%
  \pgfsetcolor{black}
  % Edge: 1 -> 1
  \draw [->] (52.683bp,42.991bp) .. controls (51.78bp,53.087bp) and (54.219bp,62.0bp)  .. (60.0bp,62.0bp) .. controls (63.704bp,62.0bp) and (66.036bp,58.342bp)  .. (67.317bp,42.991bp);
  \definecolor{strokecol}{rgb}{0.0,0.0,0.0};
  \pgfsetstrokecolor{strokecol}
  \draw (60.0bp,69.5bp) node {a};
  % Edge: 1 -> 0
  \draw [->] (82.117bp,22.0bp) .. controls (92.296bp,22.0bp) and (104.63bp,22.0bp)  .. (125.82bp,22.0bp);
  \draw (104.0bp,29.5bp) node {b};
  % Edge: 0 -> 1
  \draw [->] (128.69bp,12.134bp) .. controls (122.6bp,8.5065bp) and (115.24bp,4.8378bp)  .. (108.0bp,3.0bp) .. controls (101.24bp,1.2827bp) and (94.155bp,2.4513bp)  .. (78.473bp,9.1607bp);
  \draw (104.0bp,10.5bp) node {a};
  % Edge: 0 -> 0
  \draw [->] (136.97bp,38.664bp) .. controls (135.41bp,48.625bp) and (137.75bp,58.0bp)  .. (144.0bp,58.0bp) .. controls (148.0bp,58.0bp) and (150.4bp,54.152bp)  .. (151.03bp,38.664bp);
  \draw (144.0bp,65.5bp) node {b};
  % Edge: I1 -> 1
  \draw [->] (1.1664bp,22.0bp) .. controls (2.7183bp,22.0bp) and (14.889bp,22.0bp)  .. (37.856bp,22.0bp);
  % Node: 1
  \begin{scope}
    \definecolor{strokecol}{rgb}{0.0,0.0,0.0};
    \pgfsetstrokecolor{strokecol}
    \draw (60.0bp,22.0bp) ellipse (18.0bp and 18.0bp);
    \draw (60.0bp,22.0bp) ellipse (22.0bp and 22.0bp);
    \draw (60.0bp,22.0bp) node {1};
  \end{scope}
  % Node: 0
  \begin{scope}
    \definecolor{strokecol}{rgb}{0.0,0.0,0.0};
    \pgfsetstrokecolor{strokecol}
    \draw (144.0bp,22.0bp) ellipse (18.0bp and 18.0bp);
    \draw (144.0bp,22.0bp) node {0};
  \end{scope}
  %
\end{tikzpicture}
% End of code
          \quad
          \begin{minipage}{0.5\textwidth}
            This is the automaton sent. It has two states, one for the rows equal to 0 and one for
            the rows equal to 1. From state $1 = row(\varepsilon)$ if we read an $a$ we go rest in 1 since $row(\varepsilon \cdot a) = 1$. If we read a $b$ we go to 0 since $row(\varepsilon \cdot b) = 0$. 1 is the initial state since it is $row(\varepsilon) = 1$ and it is also an accepting state since the first bit of its row is a 1.\\
            A counter-exemple of this automaton after an equivalence query could be $aa$ since $aa$ is accepted by the A but not by the language of Teacher.
          \end{minipage}
        \end{center}
  \item  \quad
        \begin{center}
          \begin{tabular}{c || c }
            $T_3$         & $\varepsilon$ \\ [0.5ex]
            \hline\hline
            $\varepsilon$ & 1             \\
            $b$           & 0             \\
            $aa$          & 0             \\
            $a$           & 1             \\
            \hline \hline
            ba            & 1             \\
            bb            & 0             \\
            aaa           & 0             \\
            aab           & 0             \\
            ab            & 0             \\
          \end{tabular}
          \quad
          \begin{minipage}{0.5\textwidth}
            The table is closed but not consistent :\\
            $row(\varepsilon) = row(a)$ but $row(\varepsilon \cdot a) \neq row(a \cdot a)$.\\
            We can distinguish $row(\varepsilon)$ and $row(a)$ by adding $a$ to $E$ since $T(a \cdot a \cdot \varepsilon) \neq T(\varepsilon \cdot a \cdot \varepsilon)$. \\
            Note : in $a \cdot a \cdot \varepsilon$, the first $a$ is taken from $S$, the second from $\Sigma$ and $\varepsilon$ from $E$.
          \end{minipage}
        \end{center}
  \item  \quad
        \begin{center}
          \begin{tabular}{c || c | c }
            $T_4$         & $\varepsilon$ & a \\ [0.5ex]
            \hline\hline
            $\varepsilon$ & 1             & 1 \\
            $b$           & 0             & 1 \\
            $aa$          & 0             & 0 \\
            $a$           & 1             & 0 \\
            \hline \hline
            ba            & 1             & 0 \\
            bb            & 0             & 0 \\
            aaa           & 0             & 0 \\
            aab           & 0             & 0 \\
            ab            & 0             & 0 \\
          \end{tabular}
          \quad
          \begin{minipage}{0.5\textwidth}
            The table is now closed and consistent.
            It is easy to check the closedness (every line in the lower part are also in the upper part), and the consistence (there are no two similar rows in the upper part).
          \end{minipage}
        \end{center}
  \item \quad
        \begin{center}
          % Start of code
% \begin{tikzpicture}[anchor=mid,>=latex',line join=bevel,]
\begin{tikzpicture}[>=latex',line join=bevel,scale=0.5]
  \pgfsetlinewidth{1bp}
  %%
  \pgfsetcolor{black}
  % Edge: 11 -> 10
  \draw [->] (84.703bp,49.854bp) .. controls (90.624bp,51.021bp) and (97.034bp,52.168bp)  .. (103.0bp,53.0bp) .. controls (141.45bp,58.366bp) and (185.98bp,61.607bp)  .. (225.73bp,63.918bp);
  \definecolor{strokecol}{rgb}{0.0,0.0,0.0};
  \pgfsetstrokecolor{strokecol}
  \draw (156.0bp,68.5bp) node {a};
  % Edge: 11 -> 01
  \draw [->] (84.419bp,38.601bp) .. controls (95.354bp,35.409bp) and (108.83bp,31.477bp)  .. (130.96bp,25.016bp);
  \draw (107.0bp,41.5bp) node {b};
  % Edge: 10 -> 00
  \draw [->] (271.73bp,56.712bp) .. controls (285.91bp,51.179bp) and (304.82bp,43.798bp)  .. (330.36bp,33.832bp);
  \draw (299.99bp,55.5bp) node {a,b};
  % Edge: 01 -> 10
  \draw [->] (178.44bp,28.263bp) .. controls (187.59bp,32.707bp) and (198.36bp,38.023bp)  .. (208.0bp,43.0bp) .. controls (211.76bp,44.941bp) and (215.69bp,47.019bp)  .. (228.4bp,53.866bp);
  \draw (204.5bp,50.5bp) node {a};
  % Edge: 01 -> 00
  \draw [->] (183.27bp,17.347bp) .. controls (206.79bp,16.918bp) and (242.21bp,16.656bp)  .. (272.99bp,18.0bp) .. controls (287.42bp,18.63bp) and (303.27bp,19.896bp)  .. (327.09bp,22.157bp);
  \draw (249.49bp,25.5bp) node {b};
  % Edge: 00 -> 00
  \draw [->] (344.42bp,42.037bp) .. controls (342.47bp,51.858bp) and (345.66bp,61.0bp)  .. (353.99bp,61.0bp) .. controls (359.2bp,61.0bp) and (362.4bp,57.429bp)  .. (363.56bp,42.037bp);
  \draw (353.99bp,68.5bp) node {a,b};
  % Edge: I11 -> 11
  \draw [->] (1.1707bp,45.0bp) .. controls (2.7304bp,45.0bp) and (14.746bp,45.0bp)  .. (37.824bp,45.0bp);
  % Node: 11
  \begin{scope}
    \definecolor{strokecol}{rgb}{0.0,0.0,0.0};
    \pgfsetstrokecolor{strokecol}
    \draw (61.5bp,45.0bp) ellipse (19.5bp and 19.5bp);
    \draw (61.5bp,45.0bp) ellipse (23.5bp and 23.5bp);
    \draw (61.498bp,45.0bp) node {11};
  \end{scope}
  % Node: 10
  \begin{scope}
    \definecolor{strokecol}{rgb}{0.0,0.0,0.0};
    \pgfsetstrokecolor{strokecol}
    \draw (249.49bp,65.0bp) ellipse (19.5bp and 19.5bp);
    \draw (249.49bp,65.0bp) ellipse (23.5bp and 23.5bp);
    \draw (249.49bp,65.0bp) node {10};
  \end{scope}
  % Node: 01
  \begin{scope}
    \definecolor{strokecol}{rgb}{0.0,0.0,0.0};
    \pgfsetstrokecolor{strokecol}
    \draw (156.0bp,18.0bp) ellipse (27.0bp and 18.0bp);
    \draw (156.0bp,18.0bp) node {01};
  \end{scope}
  % Node: 00
  \begin{scope}
    \definecolor{strokecol}{rgb}{0.0,0.0,0.0};
    \pgfsetstrokecolor{strokecol}
    \draw (353.99bp,25.0bp) ellipse (27.0bp and 18.0bp);
    \draw (353.99bp,25.0bp) node {00};
  \end{scope}
  %
\end{tikzpicture}
% End of code
          \quad
          \begin{minipage}{0.5\textwidth}
            The automaton sent is still not valid : \\
            a counter-exemple could be $baba$ since $baba$ is accepted \\
            by the Teacher but not by the automaton.\\
            The counter-example and all of its prefixes will be added to $S$.
          \end{minipage}
        \end{center}
  \item \quad
        \begin{center}
          \begin{tabular}{c || c | c }
            $T_5$         & $\varepsilon$ & a \\ [0.5ex]
            \hline\hline
            $\varepsilon$ & 1             & 1 \\
            $b$           & 0             & 1 \\
            $aa$          & 0             & 0 \\
            $a$           & 1             & 0 \\
            $baba$        & 1             & 0 \\
            $bab$         & 0             & 1 \\
            $ba$          & 1             & 0 \\
            \hline \hline
            ba            & 1             & 0 \\
            bb            & 0             & 0 \\
            aaa           & 0             & 0 \\
            aab           & 0             & 0 \\
            ab            & 0             & 0 \\
            $babaa$       & 0             & 0 \\
            $babab$       & 0             & 1 \\
            $babb$        & 0             & 0 \\
            $baa$         & 0             & 0 \\
          \end{tabular}
          \quad
          \begin{minipage}{0.5\textwidth}
            The table is closed but not consistent :\\
            $row(a) = row(baba)$ but $row(a \cdot b) \neq row(baba \cdot b)$.\\
            We can distinguish them by adding $ba$ to $E$ since $T(a \cdot b \cdot a) \neq T(baba \cdot b \cdot a)$. \\
            Note : in $a \cdot b \cdot a$, the first $a$ is taken from $S$, the $b$ from $\Sigma$ and the second $a$ from $E$.
          \end{minipage}
        \end{center}
  \item \quad
        \begin{center}
          \begin{tabular}{c || c | c | c }
            $T_6$         & $\varepsilon$ & a & ba \\ [0.5ex]
            \hline\hline
            $\varepsilon$ & 1             & 1 & 1  \\
            $b$           & 0             & 1 & 0  \\
            $aa$          & 0             & 0 & 0  \\
            $a$           & 1             & 0 & 0  \\
            $baba$        & 1             & 0 & 1  \\
            $bab$         & 0             & 1 & 0  \\
            $ba$          & 1             & 0 & 1  \\
            \hline \hline
            ba            & 1             & 0 & 0  \\
            bb            & 0             & 0 & 0  \\
            aaa           & 0             & 0 & 0  \\
            aab           & 0             & 0 & 0  \\
            ab            & 0             & 0 & 0  \\
            $babaa$       & 0             & 0 & 0  \\
            $babav$       & 0             & 1 & 0  \\
            $babb$        & 0             & 0 & 0  \\
            $baa$         & 0             & 0 & 0  \\
          \end{tabular}
          \quad
          \begin{minipage}{0.5\textwidth}
            The table is closed and consistent.\\
            An automaton will be sent.
          \end{minipage}
        \end{center}
  \item \quad
        \begin{center}
          % Start of code
% \begin{tikzpicture}[anchor=mid,>=latex',line join=bevel,]
\begin{tikzpicture}[>=latex',line join=bevel,scale=0.5]
  \pgfsetlinewidth{1bp}
  %%
  \pgfsetcolor{black}
  % Edge: 111 -> 100
  \draw [->] (93.625bp,86.749bp) .. controls (122.46bp,93.36bp) and (168.67bp,103.96bp)  .. (210.75bp,113.6bp);
  \definecolor{strokecol}{rgb}{0.0,0.0,0.0};
  \pgfsetstrokecolor{strokecol}
  \draw (116.1bp,100.15bp) node {a};
  % Edge: 111 -> 010
  \draw [->] (93.055bp,72.884bp) .. controls (104.46bp,69.476bp) and (118.0bp,65.43bp)  .. (140.02bp,58.85bp);
  \draw (116.1bp,75.155bp) node {b};
  % Edge: 100 -> 000
  \draw [->] (265.89bp,114.94bp) .. controls (286.61bp,110.84bp) and (315.81bp,104.07bp)  .. (340.29bp,94.655bp) .. controls (354.24bp,89.288bp) and (368.94bp,81.704bp)  .. (389.98bp,69.698bp);
  \draw (312.24bp,117.15bp) node {a,b};
  % Edge: 010 -> 101
  \draw [->] (192.38bp,52.742bp) .. controls (215.21bp,53.685bp) and (248.54bp,55.063bp)  .. (284.17bp,56.536bp);
  \draw (238.14bp,64.155bp) node {a};
  % Edge: 010 -> 000
  \draw [->] (185.15bp,39.386bp) .. controls (217.44bp,20.195bp) and (284.82bp,-12.906bp)  .. (340.29bp,5.655bp) .. controls (358.02bp,11.59bp) and (374.85bp,24.079bp)  .. (395.08bp,42.464bp);
  \draw (312.24bp,13.155bp) node {b};
  % Edge: 101 -> 010
  \draw [->] (287.74bp,43.034bp) .. controls (281.03bp,39.596bp) and (273.53bp,36.421bp)  .. (266.19bp,34.655bp) .. controls (241.95bp,28.825bp) and (234.6bp,30.085bp)  .. (210.1bp,34.655bp) .. controls (205.84bp,35.449bp) and (201.46bp,36.64bp)  .. (187.56bp,41.543bp);
  \draw (238.14bp,42.155bp) node {b};
  % Edge: 101 -> 000
  \draw [->] (340.51bp,57.655bp) .. controls (350.68bp,57.655bp) and (362.39bp,57.655bp)  .. (383.28bp,57.655bp);
  \draw (361.79bp,65.155bp) node {a};
  % Edge: 000 -> 000
  \draw [->] (401.72bp,75.065bp) .. controls (400.12bp,84.743bp) and (402.97bp,93.655bp)  .. (410.29bp,93.655bp) .. controls (414.86bp,93.655bp) and (417.69bp,90.174bp)  .. (418.86bp,75.065bp);
  \draw (410.29bp,101.15bp) node {a,b};
  % Edge: I111 -> 111
  \draw [->] (1.0464bp,80.655bp) .. controls (1.8856bp,80.655bp) and (13.98bp,80.655bp)  .. (37.822bp,80.655bp);
  % Node: 111
  \begin{scope}
    \definecolor{strokecol}{rgb}{0.0,0.0,0.0};
    \pgfsetstrokecolor{strokecol}
    \draw (66.05bp,80.65bp) ellipse (24.09bp and 24.09bp);
    \draw (66.05bp,80.65bp) ellipse (28.1bp and 28.1bp);
    \draw (66.048bp,80.655bp) node {111};
  \end{scope}
  % Node: 100
  \begin{scope}
    \definecolor{strokecol}{rgb}{0.0,0.0,0.0};
    \pgfsetstrokecolor{strokecol}
    \draw (238.14bp,119.65bp) ellipse (24.09bp and 24.09bp);
    \draw (238.14bp,119.65bp) ellipse (28.1bp and 28.1bp);
    \draw (238.14bp,119.65bp) node {100};
  \end{scope}
  % Node: 010
  \begin{scope}
    \definecolor{strokecol}{rgb}{0.0,0.0,0.0};
    \pgfsetstrokecolor{strokecol}
    \draw (165.1bp,51.65bp) ellipse (27.0bp and 18.0bp);
    \draw (165.1bp,51.655bp) node {010};
  \end{scope}
  % Node: 000
  \begin{scope}
    \definecolor{strokecol}{rgb}{0.0,0.0,0.0};
    \pgfsetstrokecolor{strokecol}
    \draw (410.29bp,57.65bp) ellipse (27.0bp and 18.0bp);
    \draw (410.29bp,57.655bp) node {000};
  \end{scope}
  % Node: 101
  \begin{scope}
    \definecolor{strokecol}{rgb}{0.0,0.0,0.0};
    \pgfsetstrokecolor{strokecol}
    \draw (312.24bp,57.65bp) ellipse (24.09bp and 24.09bp);
    \draw (312.24bp,57.65bp) ellipse (28.1bp and 28.1bp);
    \draw (312.24bp,57.655bp) node {101};
  \end{scope}
  %
\end{tikzpicture}
% End of code
          \quad
          \begin{minipage}{0.5\textwidth}
            $L(A) = U \rightarrow$ the Teacher accepts the automaton.
          \end{minipage}
        \end{center}
\end{enumerate}
\subsection{NL* Execution}
\begin{enumerate}
  \item \begin{minipage}{0.3\textwidth}
          \begin{tabular}{c||c}
            $T_1$              & $\E$ \\
            \hline\hline
            *$\E$\footnotemark & 1    \\
            \hline\hline
            *b                 & 1    \\
            *a                 & 0    \\
          \end{tabular}
        \end{minipage} \footnotetext{Prime rows are indicated by a leading asterisk} \quad
        \begin{minipage}{0.6\textwidth}
          The table is not closed, because $row(a) = 0$ but there is no $s \in S$ such that $row(s) \subseteq 0$ . $a$ will be promoted.
        \end{minipage}

  \item \begin{minipage}{0.3\textwidth}
          \begin{tabular}{c||c}
            $T_2$ & $\E$ \\
            \hline\hline
            *$\E$ & 1    \\
            *a    & 0    \\
            \hline\hline
            *b    & 1    \\
            *ab   & 0    \\
            *aa   & 1    \\
          \end{tabular}
        \end{minipage}\quad
        \begin{minipage}{0.6\textwidth}
          $T_2$ is not consistent : $row(a) \sqsubseteq row(\E)$ but $row(a \cdot a) \not\sqsubseteq row(\E \cdot a)$. Column $a$ is going to be added because $T(a \cdot a \cdot \E) \neq row(\E \cdot a \cdot \E)$.
        \end{minipage}

  \item \begin{minipage}{0.3\textwidth}
          \begin{tabular}{c||c |c}
            $T_3$ & $\E$ & a \\
            \hline\hline
            *$\E$ & 1    & 0 \\
            *a    & 0    & 1 \\
            \hline\hline
            *b    & 1    & 1 \\
            *ab   & 0    & 0 \\
            *aa   & 1    & 1 \\
          \end{tabular}
        \end{minipage}\quad
        \begin{minipage}{0.6\textwidth}
          $T_3$ is not closed since $row(ab) = 00$ but $\nexists x \in S$ such that $x \sqsubseteq row(ab)$
        \end{minipage}

  \item \begin{minipage}{0.3\textwidth}
          \begin{tabular}{c||c |c}
            $T_4$ & $\E$ & a \\
            \hline\hline
            *$\E$ & 1    & 0 \\
            *a    & 0    & 1 \\
            *ab   & 0    & 0 \\
            \hline\hline
            b     & 1    & 1 \\
            aa    & 1    & 1 \\
            abb   & 1    & 1 \\
            *aba  & 0    & 0 \\
          \end{tabular}
        \end{minipage}\quad
        \begin{minipage}{0.6\textwidth}
          $T_4$ is not consistent : $row(ab) \sqsubseteq row(a)$ but $row(ab \cdot b) \not\sqsubseteq row(a \cdot b)$. Column $b$ is going to be added because $T(ab \cdot b \cdot \E) \neq row(a \cdot b \cdot \E)$.
        \end{minipage}

  \item \begin{minipage}{0.3\textwidth}
          \begin{tabular}{c||c |c|c}
            $T_5$ & $\E$ & a & b \\
            \hline\hline
            *$\E$ & 1    & 0 & 1 \\
            *a    & 0    & 1 & 0 \\
            *ab   & 0    & 0 & 1 \\
            \hline\hline
            b     & 1    & 1 & 1 \\
            aa    & 1    & 1 & 1 \\
            abb   & 1    & 1 & 1 \\
            *aba  & 0    & 0 & 0 \\
          \end{tabular}
        \end{minipage}\quad
        \begin{minipage}{0.6\textwidth}
          $T_5$ is not closed since $row(aba) = 000$ but $\nexists x \in S$ such that $x \sqsubseteq row(aba)$.\\
          $aba$ will be promoted.
        \end{minipage}

  \item \begin{minipage}{0.3\textwidth}
          \begin{tabular}{c||c |c|c}
            $T_6$ & $\E$ & a & b \\
            \hline\hline
            *$\E$ & 1    & 0 & 1 \\
            *a    & 0    & 1 & 0 \\
            *ab   & 0    & 0 & 1 \\
            *aba  & 0    & 0 & 0 \\
            \hline\hline
            b     & 1    & 1 & 1 \\
            aa    & 1    & 1 & 1 \\
            abb   & 1    & 1 & 1 \\
            *abab & 0    & 0 & 0 \\
            *abaa & 0    & 0 & 0 \\
          \end{tabular}
        \end{minipage}\quad
        \begin{minipage}{0.6\textwidth}
          $T_6$ is closed and consistent.\\
          The conjecture will be sent.
        \end{minipage}

  \item \begin{minipage}{0.3\textwidth}
          % Start of code
% \begin{tikzpicture}[anchor=mid,>=latex',line join=bevel,]
\begin{tikzpicture}[>=latex',line join=bevel,scale = 0.5]
  \pgfsetlinewidth{1bp}
  %%
  \pgfsetcolor{black}
  % Edge: 101 -> 101
  \draw [->] (57.457bp,105.29bp) .. controls (57.075bp,115.71bp) and (59.938bp,124.38bp)  .. (66.048bp,124.38bp) .. controls (69.962bp,124.38bp) and (72.544bp,120.82bp)  .. (74.639bp,105.29bp);
  \definecolor{strokecol}{rgb}{0.0,0.0,0.0};
  \pgfsetstrokecolor{strokecol}
  \draw (66.048bp,131.88bp) node {b};
  % Edge: 101 -> 010
  \draw [->] (85.821bp,58.136bp) .. controls (93.287bp,51.487bp) and (102.41bp,44.89bp)  .. (112.1bp,41.328bp) .. controls (120.17bp,38.36bp) and (129.29bp,37.23bp)  .. (148.07bp,37.247bp);
  \draw (121.1bp,48.828bp) node {b,a};
  % Edge: 101 -> 001
  \draw [->] (74.586bp,105.09bp) .. controls (85.162bp,136.95bp) and (107.78bp,188.68bp)  .. (148.1bp,211.33bp) .. controls (166.73bp,221.8bp) and (176.87bp,220.48bp)  .. (196.19bp,211.33bp) .. controls (227.7bp,196.39bp) and (248.89bp,161.35bp)  .. (264.98bp,126.03bp);
  \draw (172.14bp,225.83bp) node {b};
  % Edge: 101 -> 000
  \draw [->] (83.83bp,56.105bp) .. controls (98.536bp,38.748bp) and (121.65bp,16.142bp)  .. (148.1bp,7.3279bp) .. controls (218.57bp,-16.162bp) and (303.73bp,24.1bp)  .. (355.63bp,54.446bp);
  \draw (223.19bp,11.828bp) node {b,a};
  % Edge: 010 -> 101
  \draw [->] (149.47bp,48.944bp) .. controls (143.23bp,51.377bp) and (136.41bp,53.997bp)  .. (130.1bp,56.328bp) .. controls (121.15bp,59.631bp) and (111.45bp,63.082bp)  .. (92.89bp,69.545bp);
  \draw (121.1bp,70.828bp) node {a};
  % Edge: 010 -> 010
  \draw [->] (162.95bp,62.823bp) .. controls (161.95bp,73.258bp) and (165.01bp,82.376bp)  .. (172.14bp,82.376bp) .. controls (176.71bp,82.376bp) and (179.61bp,78.634bp)  .. (181.33bp,62.823bp);
  \draw (172.14bp,89.876bp) node {a};
  % Edge: 010 -> 001
  \draw [->] (188.1bp,58.452bp) .. controls (195.33bp,66.316bp) and (204.53bp,75.125bp)  .. (214.19bp,81.328bp) .. controls (222.44bp,86.623bp) and (232.15bp,90.907bp)  .. (250.87bp,97.482bp);
  \draw (223.19bp,96.828bp) node {b,a};
  % Edge: 010 -> 000
  \draw [->] (196.24bp,42.85bp) .. controls (221.38bp,45.652bp) and (262.74bp,50.43bp)  .. (298.29bp,55.328bp) .. controls (312.82bp,57.33bp) and (328.86bp,59.788bp)  .. (352.44bp,63.546bp);
  \draw (274.24bp,62.828bp) node {b,a};
  % Edge: 001 -> 101
  \draw [->] (250.27bp,106.62bp) .. controls (244.39bp,107.33bp) and (238.07bp,107.97bp)  .. (232.19bp,108.33bp) .. controls (194.87bp,110.59bp) and (184.87bp,113.07bp)  .. (148.1bp,106.33bp) .. controls (132.39bp,103.45bp) and (115.62bp,98.065bp)  .. (92.069bp,89.176bp);
  \draw (172.14bp,117.83bp) node {b};
  % Edge: 001 -> 010
  \draw [->] (258.28bp,85.204bp) .. controls (251.05bp,77.34bp) and (241.85bp,68.531bp)  .. (232.19bp,62.328bp) .. controls (223.94bp,57.033bp) and (214.24bp,52.749bp)  .. (195.51bp,46.174bp);
  \draw (223.19bp,69.828bp) node {b};
  % Edge: 001 -> 001
  \draw [->] (265.05bp,125.82bp) .. controls (264.05bp,136.26bp) and (267.11bp,145.38bp)  .. (274.24bp,145.38bp) .. controls (278.81bp,145.38bp) and (281.7bp,141.63bp)  .. (283.43bp,125.82bp);
  \draw (274.24bp,152.88bp) node {b};
  % Edge: 001 -> 000
  \draw [->] (297.45bp,95.338bp) .. controls (311.16bp,90.411bp) and (328.94bp,84.015bp)  .. (353.53bp,75.171bp);
  \draw (325.29bp,95.828bp) node {b,a};
  % Edge: 000 -> 000
  \draw [->] (367.15bp,89.823bp) .. controls (366.14bp,100.26bp) and (369.21bp,109.38bp)  .. (376.34bp,109.38bp) .. controls (380.9bp,109.38bp) and (383.8bp,105.63bp)  .. (385.52bp,89.823bp);
  \draw (376.34bp,116.88bp) node {b,a};
  % Edge: I101 -> 101
  \draw [->] (1.0464bp,78.328bp) .. controls (1.8856bp,78.328bp) and (13.98bp,78.328bp)  .. (37.822bp,78.328bp);
  % Edge: I001 -> 001
  \draw [->] (173.25bp,184.28bp) .. controls (175.33bp,183.25bp) and (209.05bp,166.42bp)  .. (232.19bp,147.33bp) .. controls (238.76bp,141.9bp) and (245.31bp,135.47bp)  .. (258.22bp,121.49bp);
  % Edge: I000 -> 000
  \draw [->] (275.39bp,196.16bp) .. controls (276.87bp,194.5bp) and (289.02bp,180.86bp)  .. (298.29bp,169.33bp) .. controls (318.32bp,144.39bp) and (340.22bp,115.2bp)  .. (361.47bp,86.386bp);
  % Node: 101
  \begin{scope}
    \definecolor{strokecol}{rgb}{0.0,0.0,0.0};
    \pgfsetstrokecolor{strokecol}
    \draw (66.05bp,78.33bp) ellipse (24.09bp and 24.09bp);
    \draw (66.05bp,78.33bp) ellipse (28.1bp and 28.1bp);
    \draw (66.048bp,78.328bp) node {101};
  \end{scope}
  % Node: 010
  \begin{scope}
    \definecolor{strokecol}{rgb}{0.0,0.0,0.0};
    \pgfsetstrokecolor{strokecol}
    \draw (172.14bp,40.33bp) ellipse (24.1bp and 24.1bp);
    \draw (172.14bp,40.328bp) node {010};
  \end{scope}
  % Node: 001
  \begin{scope}
    \definecolor{strokecol}{rgb}{0.0,0.0,0.0};
    \pgfsetstrokecolor{strokecol}
    \draw (274.24bp,103.33bp) ellipse (24.1bp and 24.1bp);
    \draw (274.24bp,103.33bp) node {001};
  \end{scope}
  % Node: 000
  \begin{scope}
    \definecolor{strokecol}{rgb}{0.0,0.0,0.0};
    \pgfsetstrokecolor{strokecol}
    \draw (376.34bp,67.33bp) ellipse (24.1bp and 24.1bp);
    \draw (376.34bp,67.328bp) node {000};
  \end{scope}
  %
\end{tikzpicture}
% End of code

        \end{minipage}\quad\\
        \begin{minipage}{1\textwidth}
          Construction of this automaton :
          \begin{enumerate}
            \item $Q = {101,010,001,000}$ which are the Prime rows in $S$;
            \item $Q_I = {101,001,000}$ which are all the rows that are covered by $row(\E)$;
            \item $F = {101}$ which is the only prime row having a $1$ in the column of $\E$;
            \item Transitions are more complicated to analyze, I will only describe those starting from the state $101$. We have $row(\E) = 101$ and we have to calculate the transition when reading :
                  \begin{enumerate}
                    \item $a \rightarrow row(\E \cdot a) = 010$ and the set of rows that are covered by $010$ is ${010, 000}$, so we draw two transitions labelled with $a$ from $101$ to $010$ and to $000$;
                    \item $b \rightarrow row(\E \cdot b) = 111$ and the set of rows that are covered by $111$ is ${010, 000, 101, 001}$, so we draw four transitions labelled with $a$ from $101$ to $010$, $000$, $101$ and $001$.
                  \end{enumerate}
          \end{enumerate}
          Moreover, this automaton recognizes precisely the language proposed by the Teacher, so the algorithm can stop.
        \end{minipage}
\end{enumerate}

\subsection{Example to find residual from a regular language}
In this section I would like to explain why the \textit{mDFA} of \cref{sec:l-exec} has one state more of the \textit{cRFSA} of \cref{sec:nl-exec}.

Let's start with the regular expression of $\U = \E+(b+aa+abb)(a+b)^*$. The residuals of this language can be found iteratively thanks \cref{algo:find_residuals}.


\begin{algorithm}[H]
  \caption{Algo to find residuals of a language $\U$}
  \label{algo:find_residuals}
  \KwIn{$\U$, $\Sigma$}
  \KwOut{$\RE$ = The set of residuals of $\U$}
  $\RE \gets \{\}$ \tcp*{Initialisation of $\RE$}
  $E \gets \{\E\}$\tcp*{The list of words to treat}
  \While{$E.length > 0$}{
    $current \gets E.pop()$\;
    $residual = (current)^{-1}L\footnote{This operation can be performed, for example, with the \textit{Brzozowski derivative} algorithm}$\;
    \If{$residual \notin \RE$}{
      $\RE \gets \RE \cup \{residual\}$\;
      \For{$sym \in \Sigma$}{
        $word \gets current \cdot sym$\;
        $E \gets E \cup \{word\}$\;
      }
    }
  }
  \Return $\RE$
\end{algorithm}

Let's apply this algorithm over $\U$.

\newcommand{\EM}{\varnothing}

\begin{tabular}{c|c|r|l|c|l}
  $Iteration$   & current & Operation      & Result                  & $\RE$                  & $E$                        \\
  \hline\hline
  \textit{Init} & $-$     & $-$            & $-$                     & $\RE = \{\}$           & $E = \{\E\}$               \\
  1             & $\E$    & $\E^{-1}\U$    & $\RE_1 = \U$            & $\RE = \RE \cup \RE_1$ & $E = \{a, b\}$             \\
  2             & $a$     & $a^{-1}\U$     & $\RE_2 = (a+bb)(a+b)^*$ & $\RE = \RE \cup \RE_2$ & $E = \{b, aa, ab\}$        \\
  3             & $b$     & $b^{-1}\U$     & $\RE_3 = (a+b)^*$       & $\RE = \RE \cup \RE_3$ & $E = \{aa, ab, ba, bb\}$   \\
  4             & $aa$    & $(aa)^{-1}\U$  & $(a+b)^* = \RE_3$       & $-$                    & $E = \{ab, ba, bb\}$       \\
  5             & $ab$    & $(ab)^{-1}\U$  & $\RE_4 = b(a+b)^*$      & $\RE = \RE \cup \RE_4$ & $E = \{ba, bb, aba, abb\}$ \\
  6             & $ba$    & $(ba)^{-1}\U$  & $(a+b)^* = \RE_3$       & $-$                    & $E = \{bb, aba, abb\}$     \\
  7             & $bb$    & $(bb)^{-1}\U$  & $(a+b)^* = \RE_3$       & $-$                    & $E = \{aba, abb\}$         \\
  8             & $aba$   & $(aba)^{-1}\U$ & $(a+b)^* = \RE_3$       & $-$                    & $E = \{abb\}$              \\
  9             & $abb$   & $(abb)^{-1}\U$ & $\RE_5 = \EM$           & $\RE = \RE \cup \RE_5$ & $E = \EM$                  \\
\end{tabular}

We see that the residuals of $\U$ are exactly five, the same number of state of the \textit{mDFA} returned by \textit{L*}, but only four of them are \textit{prime} because $\RE_3 = \RE_1 \cup \RE_2 \cup \RE_4$. That's why the \textit{cRFSA} has only four states.