\section{Learning with NL* }
\label{section:NL}

The NL* algorithm, described by Benedikt Bollig \textit{et al.} in \cite{NLPaper}, is an algorithm based on L* whose goal is to learn an unknown language $\U$ building potentially smaller automata then the one built by L*. To achieve this goal, NL* outputs a particular type of automaton called \textit{Canonical residual final state automata (cRFSA)}.

\subsection{Residual final state automaton (RFSA)}
\label{sec:RFSA}

% We have seen in \cref{section:L} that L* is able to create a \textit{mDFA} recognizing a regular language where each state is associated to a particular row of the \OT

\begin{theorem}
  Every regular language $L$ has a finite number of residuals.
\end{theorem}

\begin{definition}
  A residual of a language $L$ is a new language $L'$ where $L = \{\omega \cdot v, \text{ for some } \omega \in \Sigma^* \text{ and for } \forall v \in L'\}$. This is commonly noted $\omega^{-1} L = L' \text{ for some } \omega \in \Sigma^*$.
\end{definition}

\begin{example}
  \label{example:residual}
  Let $L = (aa)^*$ and $\Sigma = \{a\}$. $L$ has only two residuals $L_1 = (aa)^*$ and $L_2 = a(aa)^*$\footnotetext{A procedure to find residual of a language is proposed in \cref{algo:find_residuals} }. We can see that there does not exist any other residual because if we pose $E_{ven} = {\omega \in \Sigma^* \mid len(\omega) \equiv 0 \mod 2}$ and $O_{dd} = {\omega \in \Sigma^* \mid len(\omega) \equiv 1 \mod 2}$, we have $E_{ven} \cup O_{dd} = \Sigma^*$ and $E_{eve}^{-1}L = L_1$ and $O_{dd}^{-1}L = L_2$. No other residual is possible since there exists no other word in $\Sigma^* / \{E_{ven} \cup O_{dd}\} = \varnothing$ which can create a new residual.
\end{example}

\begin{remark}
  The set of word $\W$ over an alphabet $\Sigma$ is a syntactic monoid if for every $\omega \in \W$, $\omega^{-1}L$ gives the same residual. Moreover, each residual has a corresponding syntactic monoid.
\end{remark}

\begin{example}
  Turning back to \cref{example:residual}, we can say that $E_{ven}$ is the syntactic monoid generating $\mathcal{R_1} = (aa)^*$ and $O_{dd}$ it the other syntactic monoid generating $\mathcal{R_2} = a(aa)^*$.
\end{example}

\begin{theorem}[mDFA and Residuals]
  A DFA $\A$ is minimum if and only if, for a given language $L$, $\LA = L$ and the number of states of $A$ equals the number of the residual of $L$.
\end{theorem}

\begin{proof}
  $\rightarrow$ Let $\A = \langle \Sigma, \Q, \delta, q_i, F \rangle $ be a \textit{mDFA} and $\A_i$ the automaton $\A$ where $q_I$ is a state $q_i \in \Q$. For every $q_i \in \Q$, $\mathcal{L}(\A_i)$ is a residual $\RE_i$ of $L$. We can also define its syntactic monoid $M$ by all words $\omega$ accepted by $\A$ having $F = {q_i}$.

  $\leftarrow$ Given the set of residuals $\RE = \{\RE_1, \RE_2, \dots, \RE_n\}$ of $L$ and an alphabet $\Sigma$. For every $\RE_i$ there exists an associated monoid $M_i$ such that $\omega^{-1}\RE_i = L$ for all $\omega \in M_i$. We can construct the minimal DFA $\langle$ \alphabet{}, \states{}, \transition{}, \qzero{}, \qend{} $\rangle$ where:
  \begin{itemize}
    \item $\Q = \RE$;
    \item $\delta = \{(\RE_i, \alpha) = \RE_j \text{ for every } \RE_i, \RE_j \in \RE \text{ and } \alpha \in \Sigma \text{ if } \alpha^{-1}L(\RE_i) = L(\RE_j)\}$;
    \item $q_I$ is the residual $R_i$ such that $R_i = L$ (it is the residual whose syntactic monoid contains $\E$);
    \item $F = \{\RE_i \in \RE \text{ such that } \E \in \RE_i\}$.
  \end{itemize}
  The created automaton is minimum since for any state $q_i, q_j, \mathcal{L}(\A_i) \neq \mathcal{L}(\A_j)$.
\end{proof}

Starting from these definitions, we can see that in L* every row is associated to a state of a mDFA and therefore, since every state is associated to a residual, we can deduce by transitivity that every row of the \OT represents a residual of the language.

\begin{definition}
  An \textit{RFSA} is a \textit{non-deterministic} automaton where every state represents a residual of $L$.
\end{definition}

\begin{definition}[Prime Residual]
  Let $\RE$ be the set of residuals a language $L$, then a residual $\RE_i \in \RE$ is \textit{composed} if it exists a subset $\RE' \subseteq \{\RE - \{\RE_i\}\}$ such that $R_i = \bigcup\limits_{r \in \RE'}r$. A residual which is not composed is called \textit{prime}.
\end{definition}

\begin{definition}[Language Covering]
  A language $L'$ is covered by an other language $L$ if $L' \subseteq L$ or equivalently $\forall \omega \in \Sigma^*, \omega \in L' \rightarrow \omega \in L'$
\end{definition}

\begin{definition}
  A \textit{RFSA} is \textit{canonical} if every state is associated to a \textit{prime} residual.
\end{definition}

Let $\RE$ be the set of residual of $L$. A \textit{canonical RFSA (cRFSA)} $\A = \langle \Sigma, \Q, \delta, Q_i, F \rangle$ can be constructed as follow:
\begin{itemize}
  \item $\Q = \{\RE_i \subseteq  \RE \mid \RE_i \text{ is prime}\}$;
  \item $\delta = \{(\RE_i, \alpha) = \RE_j. \forall \RE_i, \RE_j \in \RE \text{ and } \forall \alpha \in \Sigma \text{ if } L(\RE_i) \subseteq \alpha^{-1}L(\RE_j)\}$;
  \item $Q_I = \{\RE_i \in \RE \mid \RE_i \subseteq L\}$\footnote{The initial state may not be unique, it is denoted an upper-case $\Q$};
  \item $F = \{\RE_i \in \RE \mid \E \in \RE_i\}$.
\end{itemize}

\begin{theorem}
  A \textit{cRFSA} can be exponentially smaller then a \textit{mDFA}.
\end{theorem}

\subsection{Closedness and Consistency in NL*}

The NL* algorithm has similar properties as L* since the \OT, the sets $S$, $E$, $SA$ and $S_{ext}$ and the mapping $T$ are defined as in L*. However, the closedness and consistency properties are defined in a different way.

\begin{definition}[Row Covering]
  \label{def:row-covering}
  A row $r_1$ of the \OT is covered ($\sqsubseteq$) by a row $r_2$ if every bit of $r_1$ is less or equal of the corresponding bit in $r_2$.
\end{definition}

\begin{example}
  Given $r_i = [0,1,1,0,1]$ and $r_j = [0,1,1,1,1]$, we have that $r_i \sqsubseteq r_j$.
\end{example}

\begin{definition}[Row Union]
  \label{def:row-union}
  The union ($\sqcup$) of two rows $r_1, r_2$ gives a new row $r_3$ where each bit $b_i$ is obtained as the max of the $i^{th}$ bit of $r_1$ and the $i^{th}$ bit of $r_2$
\end{definition}

\begin{example}
  Given $r_i = [0,1,0,0,1]$ and $r_j = [0,1,0,1,1]$, we have that $r_i \sqcup r_j = [0,1,1,1,1]$.
\end{example}

\begin{definition}[Composed and Prime Rows]
  A row $r$ of the \OT is composed if it exists a subset of rows $R'$ not including $r$ such that $r = \bigsqcup\limits_{i \in R'}i$
\end{definition}

\begin{notation}
  From now on the function $Prime: \omega \rightarrow {True, False}$ is a function returning $True$ if $row(\omega)$ is Prime, $False$ otherwise.
\end{notation}

\begin{definition}[Closedness]
  The \OT is closed if
  \[\forall \omega \in SA, Prime(\omega) \rightarrow \exists \omega' \in S: row(\omega) = row(\omega')\]
\end{definition}

To make the \OT close, the row $\omega$ is promoted and then the \OT is completed.

\begin{definition}[Consistency]
  The \OT is consistent if
  \[
    \forall \omega_1, \omega_2 \in S,
    row(\omega_1) \sqsubseteq row( \omega_2) \rightarrow
    \forall \alpha \in \Sigma,
    row(\omega_1 \cdot \alpha) = row(\omega_2 \cdot \alpha)
  \]
\end{definition}

If the table is not consistent, $\exists \omega_1, \omega_2 \in S \text{ such that } row(\omega_1) \sqsubseteq row(\omega_2)$ and $row(\omega_1 \cdot \alpha) \not\sqsubseteq row(\omega_2 \cdot \alpha) \text{ for an } \alpha \in \Sigma$. We can restore the consistency by making $\omega_1$ and $\omega_2$ no more RFSA-similar: we find an $e \in E$ such that $T(\omega_1 \cdot \alpha \cdot e) \sqsupset T(\omega_2 \cdot \alpha \cdot e)$ and we add $\alpha \cdot e$ in $E$.

\subsection{NL*: Automaton creation}

When the \OT is closed and consistent, the Learner can send to the Teacher a conjecture of the language that it has understood so far.

The conjecture is the automaton where:
\begin{itemize}
  \item $Q = \{row(\omega), \forall \omega \in S \text{ and } Prime(\omega) \}$;
  \item $\delta = \{ \delta(row(\omega), \alpha) = row(\omega'), \forall \omega \in S, \alpha \in \Sigma, \omega' \in S_{ext} \mid Prime(\omega) \text{ and } row(\omega') \sqsubseteq row(\omega \cdot \alpha)\}$;
  \item $\Q_I = row(\omega), \forall \omega \in S \mid Prime(\omega) \text{ and } row(\omega) \sqsubseteq row(\E) $;
  \item $F = \{row(\omega) \mid \omega \in S \text{ and } T(\omega) = 1 \}$.
\end{itemize}

After sending of the conjecture, if the Teacher answers \textit{Yes} the algorithm stops, otherwise a counter-example $\omega$ is provided. $\omega$ is added to the $E$ set and if needed $E$ is completed to remain suffix-closed.

\begin{remark}
  Notice that L* adds the counter-example in $S$ while NL* adds it in $E$.
\end{remark}

\begin{lemma}
  Let $q_I \in Q_I$ be an initial state of $\A$ and $\omega \in S_{ext}$ then $\delta(q_I, \omega) \sqsubseteq row(\omega)$.
\end{lemma}

The proof is similar to the one of \cref{lemma:L_trans_from_QI} considering that now we do not use the equality relation between states, but the covering ($\sqsubseteq$) one defined in \cref{def:row-covering}.

The word acceptance of \textit{NL*} can be proved similarly to \cref{lemma:L_acceptance}.

We can also note that:
\begin{itemize}
  \item the automaton created is no necessarily deterministic, since the transition function allows multiple successors for a given state and a given symbol;
  \item the automaton may have multiple initial states: all states whose row is covered by $row(\E)$;
  \item $row(\E)$ may not be in $Q_I$ if it is a composed row;
  \item the automaton created is the \textit{cRFSA} since the represented states are only those corresponding to a prime row.
\end{itemize}


\subsection{NL* analysis}
\subsubsection{Correcteness}
As for the L* algorithm, NL* is correct since the solution must be approved by the Teacher before stopping.

\subsubsection{Halting}
As for the L* algorithm, NL* terminates because the number of residual of the language is finite and if the conjecture is not accepted, the counter-example, will allow the Learner to find a new residual until all \textit{prime} residuals are found.

A little precision of why NL* adds the counter-example $\omega$ to $E$ instead of $S$. In general, the given counter-example belongs to a residual that is not correctly represented by the sent automaton. However, while constructing the \OT we only have partial information on the residuals, because $|E|$ may not be big enough to distinguish the between residuals. It may happen that the partial row associated to $\omega$ is not prime and therefore it will not be considered by the algorithm when testing the consistency and the closedness. On the other hand, adding $\omega$ into $E$ is a correct way find directly new information on residuals. This allows NL* to terminate and, therefore, to find the correct conjecture.

\subsubsection{Time complexity}
In the worst case, the \textit{cRFSA} has the same number of state of the \textit{mDFA} and therefore the \OT will have the same size of the \OT of L*. Again the complexity depends on the length of the Teacher's counter-example. At the end we can conclude that the time complexity is in the worst case $O(n^2 \times m)$.

However since the \textit{cRFSA} found by NL* may be smaller then the \textit{mDFA} it happens that NL* finds the solution more quickly then L* and therefore the time needed to understand $\U$ can be smaller.