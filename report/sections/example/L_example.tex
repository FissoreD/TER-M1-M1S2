\subsection{L* Execution}
\begin{enumerate}
  \item \begin{minipage}{0.3\textwidth}
          \begin{tabular}{l||l}
            $T_1$         & $\varepsilon$ \\
            \hline\hline
            $\varepsilon$ & 1             \\
            \hline\hline
            a             & 1             \\
            b             & 0             \\
          \end{tabular}
        \end{minipage} \quad
        \begin{minipage}{0.6\textwidth}
          The table is not closed, since in SA, the lower part of the table, there is $b$ where $row(b) = 0$ which is not presente in S. $b$ will be promoted.
        \end{minipage}
  \item \begin{minipage}{0.3\textwidth}
          \begin{tabular}{l||l}
            $T_2$         & $\varepsilon$ \\
            \hline\hline
            $\varepsilon$ & 1             \\
            b             & 0             \\
            \hline\hline
            a             & 1             \\
            ba            & 0             \\
            bb            & 1             \\
          \end{tabular}
        \end{minipage}\quad
        \begin{minipage}{0.6\textwidth}
          The table is closed ($\forall x \in SA, \exists s \in S | row(x) = row(s)$) and consistent (there are no two similar rows in $S$), the learner can so send its conjecture.
        \end{minipage}

  \item \begin{minipage}{0.3\textwidth}
          \begin{tikzpicture}[>=latex',line join=bevel,scale=0.5]
  \pgfsetlinewidth{1bp}
  %%
  \pgfsetcolor{black}
  % Edge: 1 -> 1
  \draw [->] (52.683bp,42.991bp) .. controls (51.78bp,53.087bp) and (54.219bp,62.0bp)  .. (60.0bp,62.0bp) .. controls (63.704bp,62.0bp) and (66.036bp,58.342bp)  .. (67.317bp,42.991bp);
  \definecolor{strokecol}{rgb}{0.0,0.0,0.0};
  \pgfsetstrokecolor{strokecol}
  \draw (60.0bp,69.5bp) node {a};
  % Edge: 1 -> 0
  \draw [->] (82.117bp,22.0bp) .. controls (92.296bp,22.0bp) and (104.63bp,22.0bp)  .. (125.82bp,22.0bp);
  \draw (104.0bp,29.5bp) node {b};
  % Edge: 0 -> 1
  \draw [->] (128.69bp,12.134bp) .. controls (122.6bp,8.5065bp) and (115.24bp,4.8378bp)  .. (108.0bp,3.0bp) .. controls (101.24bp,1.2827bp) and (94.155bp,2.4513bp)  .. (78.473bp,9.1607bp);
  \draw (104.0bp,10.5bp) node {a};
  % Edge: 0 -> 0
  \draw [->] (136.97bp,38.664bp) .. controls (135.41bp,48.625bp) and (137.75bp,58.0bp)  .. (144.0bp,58.0bp) .. controls (148.0bp,58.0bp) and (150.4bp,54.152bp)  .. (151.03bp,38.664bp);
  \draw (144.0bp,65.5bp) node {b};
  % Edge: I1 -> 1
  \draw [->] (1.1664bp,22.0bp) .. controls (2.7183bp,22.0bp) and (14.889bp,22.0bp)  .. (37.856bp,22.0bp);
  % Node: 1
  \begin{scope}
    \definecolor{strokecol}{rgb}{0.0,0.0,0.0};
    \pgfsetstrokecolor{strokecol}
    \draw (60.0bp,22.0bp) ellipse (18.0bp and 18.0bp);
    \draw (60.0bp,22.0bp) ellipse (22.0bp and 22.0bp);
    \draw (60.0bp,22.0bp) node {1};
  \end{scope}
  % Node: 0
  \begin{scope}
    \definecolor{strokecol}{rgb}{0.0,0.0,0.0};
    \pgfsetstrokecolor{strokecol}
    \draw (144.0bp,22.0bp) ellipse (18.0bp and 18.0bp);
    \draw (144.0bp,22.0bp) node {0};
  \end{scope}
  %
\end{tikzpicture}
% End of code
        \end{minipage}\quad
        \begin{minipage}{0.6\textwidth}
          This is the automaton sent.
          The states are $0$ and $1$ since there are two distinct rows in the upper part of the table, $1$ is the initial state since $row(\varepsilon) = 0$ and it is also accepting since $T(\varepsilon) = 1$.\\
          Transitions are made as follow : \\
          \begin{itemize}
            \item $\delta(1, a) = \delta(row(\varepsilon), a) = row(\varepsilon \cdot a) = 1$.
            \item $\delta(1, b) = \delta(row(\varepsilon), b) = row(\varepsilon \cdot b) =  0$.
            \item $\delta(0, a) = \delta(row(b), a) = row(b \cdot a) =  0$.
            \item $\delta(0, b) = \delta(row(b), b) = row(b \cdot b) = 1$.
          \end{itemize}
          However, the automaton is not valid, since the word \textit{ab} is not recognized by \automaton{}, bu \textit{ab} is accepted by the Teacher.
        \end{minipage}
  \item \begin{minipage}{0.3\textwidth}
          \begin{tabular}{l||l}
            $T_3$         & $\varepsilon$ \\
            \hline\hline
            $\varepsilon$ & 1             \\
            b             & 0             \\
            ab            & 1             \\
            a             & 1             \\
            \hline\hline
            ba            & 0             \\
            bb            & 1             \\
            aba           & 1             \\
            abb           & 1             \\
            aa            & 1             \\
          \end{tabular}
        \end{minipage} \quad
        \begin{minipage}{0.5\textwidth}
          \textit{ab} has been promoted in the upper part with with \textit{a} since \textit{a} was the only prefix not present in $S$.\\
          \textit{aba, abb, aa} has been added into $SA$ beacause all word of the style $s \cdot a$ where $s \in S$ and $a \in \Sigma$ must appear in $T$.\\
          This table is not consistent : $row(\varepsilon) = row(ab)$ but taking $b \in \Sigma$ gives $row(\varepsilon \cdot b) \neq row(ab \cdot b)$. Column $b$ is going to be added beacause $T(\varepsilon \cdot b \cdot \varepsilon) \neq row(ab \cdot b \cdot \varepsilon)$ (in the two cases the $\varepsilon$ before the closing parethesis is taken from $E$).
        \end{minipage}

  \item \begin{minipage}{0.3\textwidth}
          \begin{tabular}{l||l|l}
            $T_4$         & $\varepsilon$ & b \\
            \hline\hline
            $\varepsilon$ & 1             & 0 \\
            b             & 0             & 1 \\
            ab            & 1             & 1 \\
            a             & 1             & 1 \\
            \hline\hline
            ba            & 0             & 0 \\
            bb            & 1             & 1 \\
            aba           & 1             & 1 \\
            abb           & 1             & 1 \\
            aa            & 1             & 1 \\
          \end{tabular}
        \end{minipage} \quad
        \begin{minipage}{0.5\textwidth}
          Now table is not closed : $row(ba)$ is not present in the upper part of the table, so it will be promoted.
        \end{minipage}

  \item \begin{minipage}{0.3\textwidth}
          \begin{tabular}{l||l|l}
            $T_5$         & $\varepsilon$ & b \\
            \hline\hline
            $\varepsilon$ & 1             & 0 \\
            b             & 0             & 1 \\
            ab            & 1             & 1 \\
            a             & 1             & 1 \\
            ba            & 0             & 0 \\
            \hline\hline
            bb            & 1             & 1 \\
            aba           & 1             & 1 \\
            abb           & 1             & 1 \\
            aa            & 1             & 1 \\
            baa           & 0             & 0 \\
            bab           & 0             & 0 \\
          \end{tabular}
        \end{minipage} \quad
        \begin{minipage}{0.5\textwidth}
          The table is now closed and consistent.
        \end{minipage}
  \item \begin{minipage}{0.3\textwidth}
          % Start of code
% \begin{tikzpicture}[anchor=mid,>=latex',line join=bevel,]
  \begin{tikzpicture}[>=latex',line join=bevel,scale=0.5]
    \pgfsetlinewidth{1bp}
  %%
  \pgfsetcolor{black}
    % Edge: 10 -> 01
    \draw [->] (83.509bp,48.397bp) .. controls (94.723bp,53.147bp) and (108.65bp,59.044bp)  .. (130.16bp,68.155bp);
    \definecolor{strokecol}{rgb}{0.0,0.0,0.0};
    \pgfsetstrokecolor{strokecol}
    \draw (107.0bp,67.998bp) node {b};
    % Edge: 10 -> 11
    \draw [->] (84.957bp,37.022bp) .. controls (97.84bp,35.63bp) and (114.32bp,33.897bp)  .. (129.0bp,32.498bp) .. controls (153.31bp,30.182bp) and (180.79bp,27.846bp)  .. (211.84bp,25.3bp);
    \draw (148.49bp,39.998bp) node {a};
    % Edge: 01 -> 11
    \draw [->] (165.48bp,65.717bp) .. controls (177.0bp,58.67bp) and (192.85bp,48.977bp)  .. (214.99bp,35.426bp);
    \draw (189.99bp,60.998bp) node {b};
    % Edge: 01 -> 00
    \draw [->] (166.29bp,83.796bp) .. controls (178.39bp,89.779bp) and (194.96bp,97.964bp)  .. (217.73bp,109.22bp);
    \draw (189.99bp,104.0bp) node {a};
    % Edge: 11 -> 11
    \draw [->] (227.4bp,45.699bp) .. controls (226.52bp,55.998bp) and (229.22bp,64.997bp)  .. (235.49bp,64.997bp) .. controls (239.51bp,64.997bp) and (242.06bp,61.304bp)  .. (243.58bp,45.699bp);
    \draw (235.49bp,72.497bp) node {a,b};
    % Edge: 00 -> 00
    \draw [->] (227.64bp,135.63bp) .. controls (226.17bp,145.71bp) and (228.79bp,155.0bp)  .. (235.49bp,155.0bp) .. controls (239.79bp,155.0bp) and (242.4bp,151.19bp)  .. (243.35bp,135.63bp);
    \draw (235.49bp,162.5bp) node {a,b};
    % Edge: I10 -> 10
    \draw [->] (1.1707bp,39.498bp) .. controls (2.7304bp,39.498bp) and (14.746bp,39.498bp)  .. (37.824bp,39.498bp);
    % Node: 10
  \begin{scope}
    \definecolor{strokecol}{rgb}{0.0,0.0,0.0};
    \pgfsetstrokecolor{strokecol}
    \draw (61.5bp,39.5bp) ellipse (19.5bp and 19.5bp);
    \draw (61.5bp,39.5bp) ellipse (23.5bp and 23.5bp);
    \draw (61.498bp,39.498bp) node {10};
  \end{scope}
    % Node: 01
  \begin{scope}
    \definecolor{strokecol}{rgb}{0.0,0.0,0.0};
    \pgfsetstrokecolor{strokecol}
    \draw (148.49bp,75.5bp) ellipse (19.5bp and 19.5bp);
    \draw (148.49bp,75.498bp) node {01};
  \end{scope}
    % Node: 11
  \begin{scope}
    \definecolor{strokecol}{rgb}{0.0,0.0,0.0};
    \pgfsetstrokecolor{strokecol}
    \draw (235.49bp,23.5bp) ellipse (19.5bp and 19.5bp);
    \draw (235.49bp,23.5bp) ellipse (23.5bp and 23.5bp);
    \draw (235.49bp,23.498bp) node {11};
  \end{scope}
    % Node: 00
  \begin{scope}
    \definecolor{strokecol}{rgb}{0.0,0.0,0.0};
    \pgfsetstrokecolor{strokecol}
    \draw (235.49bp,117.5bp) ellipse (19.5bp and 19.5bp);
    \draw (235.49bp,117.5bp) node {00};
  \end{scope}
  %
  \end{tikzpicture}
  % End of code
  
        \end{minipage}\quad
        \begin{minipage}{0.6\textwidth}
          This automaton recognized perfectly the language proposed by the Teacher, so the algorithm can stop.
        \end{minipage}
\end{enumerate}
% \begin{enumerate}
%   \item \quad
%         \begin{center}
%           \begin{tabular}{l||l}
%             $T_1$         & $\varepsilon$ \\ \hline\hline
%             $\varepsilon$ & 0             \\
%             \hline \hline
%             a             & 0             \\
%             b             & 0             \\
%           \end{tabular}
%           \quad
%           \begin{minipage}{0.5\textwidth}
%             The table is closed and consistent.\\
%             An automaton will be sent.
%           \end{minipage}
%         \end{center}
%   \item \quad
%         \begin{center}
%           \begin{tikzpicture}[>=latex',line join=bevel,scale=0.5]
  \pgfsetlinewidth{1bp}
  %%
  \pgfsetcolor{black}
  % Edge: 1 -> 1
  \draw [->] (52.683bp,42.991bp) .. controls (51.78bp,53.087bp) and (54.219bp,62.0bp)  .. (60.0bp,62.0bp) .. controls (63.704bp,62.0bp) and (66.036bp,58.342bp)  .. (67.317bp,42.991bp);
  \definecolor{strokecol}{rgb}{0.0,0.0,0.0};
  \pgfsetstrokecolor{strokecol}
  \draw (60.0bp,69.5bp) node {a};
  % Edge: 1 -> 0
  \draw [->] (82.117bp,22.0bp) .. controls (92.296bp,22.0bp) and (104.63bp,22.0bp)  .. (125.82bp,22.0bp);
  \draw (104.0bp,29.5bp) node {b};
  % Edge: 0 -> 1
  \draw [->] (128.69bp,12.134bp) .. controls (122.6bp,8.5065bp) and (115.24bp,4.8378bp)  .. (108.0bp,3.0bp) .. controls (101.24bp,1.2827bp) and (94.155bp,2.4513bp)  .. (78.473bp,9.1607bp);
  \draw (104.0bp,10.5bp) node {a};
  % Edge: 0 -> 0
  \draw [->] (136.97bp,38.664bp) .. controls (135.41bp,48.625bp) and (137.75bp,58.0bp)  .. (144.0bp,58.0bp) .. controls (148.0bp,58.0bp) and (150.4bp,54.152bp)  .. (151.03bp,38.664bp);
  \draw (144.0bp,65.5bp) node {b};
  % Edge: I1 -> 1
  \draw [->] (1.1664bp,22.0bp) .. controls (2.7183bp,22.0bp) and (14.889bp,22.0bp)  .. (37.856bp,22.0bp);
  % Node: 1
  \begin{scope}
    \definecolor{strokecol}{rgb}{0.0,0.0,0.0};
    \pgfsetstrokecolor{strokecol}
    \draw (60.0bp,22.0bp) ellipse (18.0bp and 18.0bp);
    \draw (60.0bp,22.0bp) ellipse (22.0bp and 22.0bp);
    \draw (60.0bp,22.0bp) node {1};
  \end{scope}
  % Node: 0
  \begin{scope}
    \definecolor{strokecol}{rgb}{0.0,0.0,0.0};
    \pgfsetstrokecolor{strokecol}
    \draw (144.0bp,22.0bp) ellipse (18.0bp and 18.0bp);
    \draw (144.0bp,22.0bp) node {0};
  \end{scope}
  %
\end{tikzpicture}
% End of code
%           \quad
%           \begin{minipage}{0.5\textwidth}
%             This is the automaton sent. It has one state since there is only one line in the upper part. From state $0 = row(\varepsilon)$ if we read an $a$ we go to state $0$ since $row(\varepsilon \cdot a) = 0$. If we read a $b$ we go to $0$ since $row(\varepsilon \cdot b) = 0$. $0$ is the initial state since $row(\varepsilon) = 0$.\\
%             There is no accepting state, because there is no row having a $1$ in the $\varepsilon$ column.\\
%             A counter-exemple of this automaton after an equivalence query could be $aa$ since $aa$ is accepted by the Teacher but not by the conjecture.
%           \end{minipage}
%         \end{center}
%   \item  \quad
%         \begin{center}
%           \begin{tabular}{c || c }
%             $T_2$         & $\varepsilon$ \\ \hline\hline
%             $\varepsilon$ & 0             \\
%             aa            & 1             \\
%             a             & 0             \\
%             \hline\hline
%             b             & 0             \\
%             aaa           & 1             \\
%             aab           & 1             \\
%             ab            & 1             \\
%           \end{tabular}
%           \quad
%           \begin{minipage}{0.5\textwidth}
%             The table is closed but not consistent :\\
%             $row(\varepsilon) = row(a)$ but $row(\varepsilon \cdot a) \neq row(a \cdot a)$.\\
%             We can distinguish $row(\varepsilon)$ and $row(a)$ by adding $a$ to $E$ since $T(a \cdot a \cdot \varepsilon) \neq T(\varepsilon \cdot a \cdot \varepsilon)$. \\
%             Note : in $a \cdot a \cdot \varepsilon$, the first $a$ is taken from $S$, the second from $\Sigma$ and $\varepsilon$ from $E$.
%           \end{minipage}
%         \end{center}
%   \item  \quad
%         \begin{center}
%           \begin{tabular}{c || c | c }
%             $T_3$         & $\varepsilon$ & a \\ \hline\hline
%             $\varepsilon$ & 0             & 0 \\
%             aa            & 1             & 1 \\
%             a             & 0             & 1 \\
%             \hline\hline
%             b             & 0             & 0 \\
%             aaa           & 1             & 1 \\
%             aab           & 1             & 0 \\
%             ab            & 1             & 0 \\
%           \end{tabular}
%           \quad
%           \begin{minipage}{0.5\textwidth}
%             The table is not closed since $row(aab) = 10$ is not present in the upper part of the table.\\
%             $aab$ is going to be promoted and $aaba$ and $aabb$ are going to be add into $SA$.
%           \end{minipage}
%         \end{center}
%   \item \quad
%         \begin{center}
%           \begin{tabular}{l||l|l}
%             $T_4$         & $\varepsilon$ & a \\ \hline\hline
%             $\varepsilon$ & 0             & 0 \\
%             aa            & 1             & 1 \\
%             a             & 0             & 1 \\
%             aab           & 1             & 0 \\ \hline\hline
%             b             & 0             & 0 \\
%             aaa           & 1             & 1 \\
%             ab            & 1             & 0 \\
%             aaba          & 0             & 1 \\
%             aabb          & 0             & 0 \\
%           \end{tabular}
%           \quad
%           \begin{minipage}{0.5\textwidth}
%             The table is closed and consistent.\\
%             An automaton will be sent.
%           \end{minipage}
%         \end{center}
%   \item \quad
%         \begin{center}
%           % Start of code
% \begin{tikzpicture}[anchor=mid,>=latex',line join=bevel,]
  \begin{tikzpicture}[>=latex',line join=bevel,scale=0.5]
    \pgfsetlinewidth{1bp}
  %%
  \pgfsetcolor{black}
    % Edge: 10 -> 01
    \draw [->] (83.509bp,48.397bp) .. controls (94.723bp,53.147bp) and (108.65bp,59.044bp)  .. (130.16bp,68.155bp);
    \definecolor{strokecol}{rgb}{0.0,0.0,0.0};
    \pgfsetstrokecolor{strokecol}
    \draw (107.0bp,67.998bp) node {b};
    % Edge: 10 -> 11
    \draw [->] (84.957bp,37.022bp) .. controls (97.84bp,35.63bp) and (114.32bp,33.897bp)  .. (129.0bp,32.498bp) .. controls (153.31bp,30.182bp) and (180.79bp,27.846bp)  .. (211.84bp,25.3bp);
    \draw (148.49bp,39.998bp) node {a};
    % Edge: 01 -> 11
    \draw [->] (165.48bp,65.717bp) .. controls (177.0bp,58.67bp) and (192.85bp,48.977bp)  .. (214.99bp,35.426bp);
    \draw (189.99bp,60.998bp) node {b};
    % Edge: 01 -> 00
    \draw [->] (166.29bp,83.796bp) .. controls (178.39bp,89.779bp) and (194.96bp,97.964bp)  .. (217.73bp,109.22bp);
    \draw (189.99bp,104.0bp) node {a};
    % Edge: 11 -> 11
    \draw [->] (227.4bp,45.699bp) .. controls (226.52bp,55.998bp) and (229.22bp,64.997bp)  .. (235.49bp,64.997bp) .. controls (239.51bp,64.997bp) and (242.06bp,61.304bp)  .. (243.58bp,45.699bp);
    \draw (235.49bp,72.497bp) node {a,b};
    % Edge: 00 -> 00
    \draw [->] (227.64bp,135.63bp) .. controls (226.17bp,145.71bp) and (228.79bp,155.0bp)  .. (235.49bp,155.0bp) .. controls (239.79bp,155.0bp) and (242.4bp,151.19bp)  .. (243.35bp,135.63bp);
    \draw (235.49bp,162.5bp) node {a,b};
    % Edge: I10 -> 10
    \draw [->] (1.1707bp,39.498bp) .. controls (2.7304bp,39.498bp) and (14.746bp,39.498bp)  .. (37.824bp,39.498bp);
    % Node: 10
  \begin{scope}
    \definecolor{strokecol}{rgb}{0.0,0.0,0.0};
    \pgfsetstrokecolor{strokecol}
    \draw (61.5bp,39.5bp) ellipse (19.5bp and 19.5bp);
    \draw (61.5bp,39.5bp) ellipse (23.5bp and 23.5bp);
    \draw (61.498bp,39.498bp) node {10};
  \end{scope}
    % Node: 01
  \begin{scope}
    \definecolor{strokecol}{rgb}{0.0,0.0,0.0};
    \pgfsetstrokecolor{strokecol}
    \draw (148.49bp,75.5bp) ellipse (19.5bp and 19.5bp);
    \draw (148.49bp,75.498bp) node {01};
  \end{scope}
    % Node: 11
  \begin{scope}
    \definecolor{strokecol}{rgb}{0.0,0.0,0.0};
    \pgfsetstrokecolor{strokecol}
    \draw (235.49bp,23.5bp) ellipse (19.5bp and 19.5bp);
    \draw (235.49bp,23.5bp) ellipse (23.5bp and 23.5bp);
    \draw (235.49bp,23.498bp) node {11};
  \end{scope}
    % Node: 00
  \begin{scope}
    \definecolor{strokecol}{rgb}{0.0,0.0,0.0};
    \pgfsetstrokecolor{strokecol}
    \draw (235.49bp,117.5bp) ellipse (19.5bp and 19.5bp);
    \draw (235.49bp,117.5bp) node {00};
  \end{scope}
  %
  \end{tikzpicture}
  % End of code
  
%           \quad
%           \begin{minipage}{0.5\textwidth}
%             $L(A) = U \rightarrow$ the Teacher accepts the automaton.
%           \end{minipage}
%         \end{center}
% \end{enumerate}