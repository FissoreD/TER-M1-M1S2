\section{Learning with L* }
\label{section:L}

The \textit{L*} algorithm, described by Dana Angluin in \cite{LPaper}, is a learning algorithm whose goal is to build a \textit{mDFA} recognizing an unknown language $\U$ thanks to membership and equivalence queries. This operation is done with some important bricks that should be considered taken into account to let the algorithm work. Some of these bricks will also be reused in by the \textit{NL*} algorithm.

\subsection{The observation table (O.T.)}
In previous section, we have talked about how \textit{Learner} and \textit{Teacher} interact, but it is also very important that every time the \textit{Learner} received an answer is stored in memory to construct little by little our overall knowledge to find, a day, the right automaton $\A$. To do this every piece of information obtained is immediately added to a particular table called the observation table (noted \OT).

\begin{definition}[Observation Table]
  The observation table is the memory of the \textit{Learner} where rows are indexed by the elements of a non-empty set of words called $S$, columns by elements of another non-empty set of words called $E$. We also have a mapping $T : \omega \rightarrow \{0, 1\}$ and the cell $c_{i,j}$ of the \OT is given by $T(S[i] \cdot E[j])$\footnote{$S[i]$ (resp. $E[j]$) indicates the $i^{th}$ (resp $j^{th}$) element of $S$ (resp. $E$). Even if sets have no order, we consider for simplicity that the $i^{th} $ element of the set is the $i^{th}$ element added in it}.
\end{definition}

$S$ (resp. $E$) has an important property, that will be discussed next, since it is \textit{prefix-closed} (resp. \textit{suffix-closed}).

\begin{definition}[Prefix (resp. Suffix)]
  A prefix (resp. suffix) of a word $\omega$, noted $Pref(\omega)$ (resp. $Suff(\omega)$), is a word $\omega_1 \in \Sigma^*$ such that $\exists \omega_2 \in \Sigma^*$ where $\omega_1 \cdot \omega_2 = \omega$ (resp. $\omega_2 \cdot \omega_1 = \omega$).
\end{definition}

\begin{definition}[Prefix-closedness (resp. Suffix-closedness)]
  A non-empty set $S$ of words is \textit{prefix-closed} if
  \[\forall \omega \in S, \forall Pref(\omega), Pref(\omega) \in S\]
  and suffix-closed if
  \[\forall \omega \in S, \forall Suff(\omega), Suff(\omega) \in S\]
  In both cases, $S$ should contain $\E$.
\end{definition}


The \OT table must be completed by all word on the form $\omega \cdot \alpha$ for $\forall \omega \in S$ and $\forall \alpha \in \Sigma$ if $\omega \cdot \alpha \notin S$. This set of new words are added to the rows of the \OT and stored in a set called $SA$.
\[SA = \{\forall s \in S, \forall c \in \Sigma : s \cdot c \mid s \cdot c \notin S \} \]
We also note $S_{ext}$ to be $S_{ext} = S \cup SA$.

\begin{example}
  If $S = \{\E, a, aa\}$ and $\Sigma = \{a, b\}$ then
  \begin{align*}
    SA      & = \left\{ b, ab, aaa, aab \right\}\footnote{$b$ is in $SA$ since it is the concatenation of $\E \in S$ and $b \in Sigma$}
    \\
    S_{ext} & = \left\{\E, a, aa, b, ab, aaa, aab\right\}
  \end{align*}
\end{example}

To make the \OT more clear, we will divide it into two parts : the \upperPart containing rows indexed by elements of $S$ and the \lowerPart containing rows indexed by element of $SA$.

\begin{notation}[row function]
  For every $\omega$ in $S_{ext}$, $row(\omega)$ is the bit-vector made buy the concatenation of all $T(\omega \cdot \omega')$ for $\forall \omega' \in E$.
\end{notation}

\begin{remark}
  Adding a word $\omega$ in $S_{ext}$ where $\omega$ is at position $i$ means that every cell $c_{i,j}$ for $\forall 0 \leq j \leq |E|$\footnote{$|E|$ is the cardinal of E, that is the number of element of this set}, we send a membership query to the Teacher and we store the answer in the corresponding cell of the \OT
\end{remark}

\subsection{Closedness and Consistence in L*}

In both algorithms, the \OT should respect the closedness and the consistence properties but the definition of them varies from the two algorithms. In this section we are going to define these properties for the L* algorithm.

\begin{definition}[Closedness]
  The \OT is closed if every row of the \lowerPart is also present in the \upperPart.
  \[\forall \omega \in SA, \exists \omega' \in S : row(\omega) = row(\omega')\]
\end{definition}

\begin{definition}[Row Similarity] Two row $r_1, r_2$ are similar if they have same corresponding bit-vector.
\end{definition}

\begin{definition}[Row Promotion]
  \label{def:rowProm}
  Given an word $\omega \in SA$ at position, then $row(\omega)$, or equivalently $\omega$ is promoted when $\omega$ is moved from $SA$ to $S$. After a promotion, the \OT must be completed again.
\end{definition}

\begin{lemma}
  \label{lemmaPromoteLinePrefixClosedness}
  If we move an element from $SA$ to $S$ then $S$ remains \textit{prefix-closed}.
\end{lemma}

\begin{proof}
  Every element $\omega$ of $SA$ is the concatenation of an element $\omega' \in S$ and a letter $\alpha \in \Sigma$. We have that $Pref(\omega) = Pref(\omega' \cdot a) = \{\omega' \cdot a\} \cup Pref(\omega')$. By definition, $S$ is prefix-closed, so if $\omega \in S$ every $Pref(\omega')$ is already in $S$. Therefore adding $\omega$ in $S$ will not impact the prefix-closedness of $S$.
\end{proof}

If the table is not closed then $\exists \omega \in SA$ such that $\nexists \omega' \in S : row(\omega) = row(\omega')$. To make the \OT closed we promote $row(\omega)$ and then, as said in \cref{def:rowProm}, we complete again the \OT

\begin{definition}[Consistence]
  The \OT is consistent if for every $\omega_1,\omega_2 \in S$, if $row(\omega_1) = row(\omega_2)$ then for every $\alpha \in \Sigma$, $row(\omega_1 \cdot \alpha) = row(\omega_2 \cdot \alpha)$.
  \[\forall \omega_1,\omega_2 \in S, \forall \alpha \in \Sigma :  row(\omega_1) = row(\omega_2) \rightarrow row(\omega_1 \cdot \alpha) = row(\omega_2 \cdot \alpha)\]
\end{definition}

If the table is not consistent, there exists $row(\omega_1) = row(\omega_2)$ and $row(\omega_1 \cdot \alpha) \neq row(\omega_2 \cdot \alpha)$ for some $\alpha \in \Sigma \text{ and some }\omega_1, \omega_2 \in S$. We can restore the consistence property by making $\omega_1$ and $\omega_2$ no more similar. To do this, we can add a new column in the \OT in such a way that the new bit $b_1$ added in $row(\omega_1)$ is different from the new bit $b_2$ added in $row(\omega_2)$. To be sure to find the right column to add, we look for an $e \in E$ such that $T(\omega_1 \cdot \alpha \cdot e) \neq T(\omega_2 \cdot \alpha \cdot e)$ and we add $\alpha \cdot e$ in $E$. In this way we are sure that $b_1 = T(\omega_1 \cdot \alpha \cdot e)$ will be different from $b_2 = T(\omega_2 \cdot \alpha \cdot e)$. Is in this moment that we can understand utility of the additional set $SA$. When we look for $T(\omega_1 \cdot \alpha \cdot e)$ (similarly for the case of $\omega_2$), we are guaranteed that $\omega_1 \cdot \alpha \cdot e$ exists in $T$ since, $\omega_1$ being in $S$, we are sure that $\omega_1' = \omega_1 \cdot \alpha \in S_{ext}$ (either $\omega_1' \in S$ or $\omega_1' \in SA$). So $T(\omega_1 \cdot \alpha \cdot e) = T(\omega_1' \cdot e)$ exists.

\begin{lemma}
  If we add a word $\omega = \alpha \cdot e$ where $\alpha \in \Sigma \text{ and } e \in E$ then $E$ remains \textit{suffix-closed}.
\end{lemma}

The proof is analogue to the proof of \cref{lemmaPromoteLinePrefixClosedness}.

\subsection{L* : Automaton creation}

When the observation table is closed and consistent, the Learner can send to Teacher a conjecture of the language the it has understood so far.

The conjecture is made on the form of an automaton where :
\begin{itemize}
  \item $Q = \{row(\omega), \forall \omega \in S\}$;
  \item $\delta(row(\omega), \alpha) = \{row(\omega \cdot \alpha), \forall \omega \in S, \alpha \in \Sigma\}$\footnote{Again we know that $\omega' = \omega\cdot\alpha$ exists in $S_{ext}$ since at worst $\omega'$ is in $SA$, the table being complete };
  \item $q_I = row(\E)$;
  \item $F = \{row(\omega) \mid \omega \in S \text{ and } T(\omega) = 1 \}$.
\end{itemize}

After the sending of the conjecture, if the Teacher answers \textit{Yes} the algorithm stops, otherwise a counter-example $\omega$ is provided. $\omega$ and all of its prefixes are to $S$ and again the \OT is completed.

\begin{lemma}
  \label{lemma:L_trans_from_QI}
  Let $q_I$ be the initial state and $s $ a word in $S_{ext}$ then $\delta(q_I, s) = row(s)$.
\end{lemma}

\begin{proof}
  We prove this by induction.\\
  Let's take a word $w$ over $S_{ext}$ and let $q_I$ be the initial state. \\
  If $length(w) = 0$, then $\delta(q_I, \E) = \delta(row(\E), \E) =  row(\E \cdot \E) = row(\E) = q_I$ that is true by definition.\\
  Let's suppose this property true for every word $w_n \in S_{ext}$ of length $n$, we want to prove that $\delta(q_I, w_{n+1}) = row(w_{n+1})$ for a word $w \in S_{ext}$ of length $n+1$.\\
  Let $w_{n+1}$ be a word of length $n+1$ and let $w = w_n \cdot x$ where $x \in \Sigma$ and $w_n$ is the prefix of $w_{n+1}$ with length $n$. Since $S_{ext}$ is prefix closed, then $w_n$ should exists.\\
  We have $\delta(q_I, w_{n+1}) = \delta(q_I, w_n \cdot x) = \delta(\delta(q_I, w_n), x)$. By the induction hypothesis, $\delta(q_I, w_n)$ exists since $length(w_n) = n$ and is equal to $row(w_n)$, therefore $\delta(\delta(q_I, w_n), x) = \delta(row(w_n), x)$. \\
  The \OT is closed so $\exists s \in S$ such that $row(w_n) = row(s)$ and so $\delta(row(w_n), x) = \delta(row(s), x)$.\\
  By definition of the transition function, $\delta(row(s), x) = row(s \cdot x)$ and by construction of the $O.T$, for every $s \in S$ there exists an element in $S_{ext}$ equals to the concatenation of this word and a letter of $\Sigma$.\\
  We can conclude that $\delta(q_I, w_{n+1}) = \delta(row(s), x) = row(s \cdot x)$, and, again, since the table is closed, $\exists s' \in S$ such that $row(s \cdot x) = row(s')$ and since every row of the upper-part is linked to a state $q_i$ of \automaton{} and this state correspond the ending state of the run of the word $w_{n+1}$.
\end{proof}

\begin{lemma}
  \label{lemma:L_acceptance}
  Let $s \in S_{ext} \text{ and } e \in E$, then $\delta(q_I, s \cdot e)$ is final if $T(s \cdot e) = 1$.
\end{lemma}

\begin{proof}
  As previous lemma, we prove this by induction on the length of the word $e$.\\
  If $length(e) = 0$, then $e = \E$ and by definition, $row(s \cdot \E) = row(s)$ is acceptant if $T(s) = 1$.\\
  Let's suppose this property true for every word $e_n$ of length $n$.\\
  Let $e_{n+1} \in E$ be a word of length $n + 1$, $e_n \in E$ the suffix of $e_{n+1}$ of length $n - 1$ and $x \in \Sigma$ such that $x \cdot e_n = e_{n+1}$. Since $E$ is suffix close, we know that $e_n$ should exist in $E$.\\
  We have that $\delta(q_I, s \cdot e_{n+1}) = \delta(q_I, s \cdot x \cdot e_n) = \delta(\delta(q_I, s), x \cdot e_n) = \delta(row(s), x \cdot e_n) = \delta(\delta(row(s), x), e_n)$.\\
  Since the table is closed, $\exists s' \in S$ such that $row(s) = row(s')$, and so $\delta(row(s), x) = \delta(row(s'), x) = row(s' \cdot x)$ since, as for the preceding lemma. Again, since the table is closed, $\exists s'' \in S$ such that $row(s' \cdot x) = row(s'')$.\\
  We have so that $\delta(q_I, s \cdot e_{n+1}) = \delta(row(s''), e_n) = \delta(\delta(q_I, s''), e_n) = \delta(q_I, s'' \cdot e_n)$.\\
  Since $s'' \in S_{ext}$ and $e_n$ has length $n$, we can prove that  $\delta(q_I, s \cdot e_{n+1})$ is in acceptant if $T(s \cdot e_{n+1}) = T(s'' \cdot e_n) = 1$.
\end{proof}

\begin{lemma}
  The automaton created is deterministic.
\end{lemma}

\begin{proof}
  We know that for every row $s_1 \in S$ and for every $x \in \Sigma$, $s_1 \cdot x \in S_{ext}$. Moreover, since the table is consistent, for every $s_2 \in S$ where $row(s_1)$ and $row(s_2)$ are similar, we have that $row(s_1 \cdot x) = row(s_2 \cdot x)$. Therefore there exists only one successor for every state $q \in Q$ and every letter $x \in \Sigma$.
\end{proof}

\begin{lemma}
  The automaton created is minimum.
\end{lemma}

\begin{proof}
  We can prove this lemma by contradiction. Let suppose that $A$ is not minimal and, therefore, that it is possible to merge \footnote{Two states can be merged if they recognize the same language or as in this case, they have same incoming and outgoing arrows} some states $q_i, q_j$ to minimize it.
  By construction, $q_i$ and $q_j$ are linked to two different rows $r_1, r_2$ of the observation table.
  Let $e \in E$ be a column which is accepted by only one between $r_1, r_2$.
  Then it means that $e$ is accepted by only one of the residual languages recognized by $q_i$ and $q_j$ and therefore we cannot merge them.
  We have proved that it is not possible to merge any couple of state $q_i, q_j \in Q$ and so the automaton is minimum.
\end{proof}

\subsection{L* analysis}
\subsubsection{Correcteness}
The goal of the algorithm is to find an automaton recognizing the language of the teacher. The solution found by the Learner is correct since it has been accepted by the Teacher.
\subsubsection{Terminaison}
The algorithm terminates only if the Teacher answer $Yes$ to a \textit{equivalence query} of the Learner.

We know, by definition, that the language $U$ known by the teacher is regular. It means that the number $n$ of its residuals is finite implying that the number of states of the minimum DFA recognizing it and the number of different rows in the $O.T$ is finite (and equal to $n$).

If in a certain moment we have $n' < n$ different rows and the table is closed and consistent, the Learner make its conjecture and sends it to the Teacher. This conjecture cannot be the good one, and the counter-example provided will have an associated row similar to a row in $S$ but which makes the \OT no more consistent. The Learner repeats the algorithm, and every time the table is not consistent, the $n'$ will increase of one. When $n' = n$ the conjecture will have $n$ states, the same of the minimum DFA, and therefore since a minimum DFA is unique, the teacher will answer $Yes$ to the last conjecture allowing the algorithm to stop.

\subsubsection{Time complexity}
Let suppose that the \cref{L_star_algo} takes constant time to check if the table is consistent and closed and to create an automaton from the $O.T$ and pose $A$ the minimum DFA recognizing $U$.

The time complexity should be measured on the number of cells of the $O.T$.

In the worst case, the number of columns $n$ (that is the cardinal of $E$) equals the number of states $A$, since at most we should have to make the table consistent $n-1$ times.

The size of $S$ is another key factor for the time complexity. Its size really depends on the length $m$ of the counter-example provided by the Teacher. At least $S$ has $n$ rows, one per number of column, but considering the length of $m$, and knowing that all prefix of $m$ will be added to $S$, then $|S|$ will have $O(n \times m)$ rows.

The size of $SA$ is at most $O(|\Sigma|)$ times the length $|S|$ since for every word in $|S|$ we can potentially have $|\Sigma|$ successors to add in $SA$.

In conclusion we have $O(n)$ columns and $O(|\Sigma| \times n \times m)$ rows, asking an overall complexity of $O(n \times |\Sigma| \times n \times m) = O(|\Sigma| \times n^2 \times m) = O(n^2 \times m)$ since $|\Sigma|$ is constant. The time complexity is so polynomial on the size of the automaton and the larger counter-example given by the teacher.