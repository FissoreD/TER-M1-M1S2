\section{Before Learning}
As said in \cref{sec:intro}, the goal of \textit{L*} and \textit{NL*} is to understand a regular an unknown language $\U$ and output a conjecture which, when accepted, will be an automaton $\A$ recognizing $\U$.

Both algorithms work on the idea that a \textit{Learner} interacts with a \textit{Teacher} which knows $\U$ via two kinds of communication.

\begin{definition}[Membership query]
  A \textit{membership query} is a query where the \textit{Learner} asks the \textit{Teacher} if a word $\omega$ belongs to $\U$.
\end{definition}

\begin{definition}[Equivalence query]
  An \textit{equivalence query} is a query where the \textit{Learner} sends an Automaton (representing a conjecture) to the \textit{Teacher} which can answer \textit{Yes} if $\LA = \U$, otherwise a word $\omega$ belonging to $\U$ but not to $\LA$ or vice versa. In this case, $\omega$ is called \textit{counter-example}.
\end{definition}

A counter-example can also be defined as a word $\omega$ belonging to the symmetrical difference between ($\Delta$) $\LA$ and $\U$.
\[\LA \Delta \U \equiv (\LA \cap \overline{\U}) \cup (\overline{\LA} \cap \U)\]

Mathematically:
\begin{equation}
  \begin{aligned}
    \label{eq:interactions}
    member(\omega) & =
    \begin{cases}
      \text{True}  & \quad  \text{if } \omega \in \U \\
      \text{False} & \quad   \text{otherwise}
    \end{cases}
    \\
    equiv(\A)      & =
    \begin{cases}
      \text{Yes}               & \quad \text{if } \LA = \U \\
      \omega \in \U \Delta \LA & \quad \text{otherwise}
    \end{cases}
  \end{aligned}
\end{equation}
