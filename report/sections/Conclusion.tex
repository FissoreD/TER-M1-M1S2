\section{Conclusion}

In this paper, I have been able to talk about some ideas on how regular languages can be learnt in different way. We are able to represent them via both minimum deterministic finite state automata or canonical residual finite state automata. The first type of automaton is maybe more understandable for humans: it is easy check if a word is accepted by the automaton, but on the other side, it can be much expensive in space.

Other studies have been done to seek other way to create automaton which can be even smaller then \textit{cRFSA}s, for example we can have automata that uses logical operators on transitions such as the \textit{Alternating Finite State Automata} or other automata using links between symbols of the alphabet.

All of these algorithms are all an evolution of the \cite{LPaper} but work on languages which are regular. It should be interesting to see if it is possible to conceive learning algorithms for not regular languages such as context free grammars. Of course this implementations would have to output more expressive classes of automata such as the automata working with an auxiliary stack. It is clear, however, that the more complicated the language is, the more powerful will be the returned data-structure representing it.

% These algorithms

\subsection{Thanks}
I would like to thanks Ms. Di Giusto and M. Lozes for their support during the development of this study, they have been really helpful to suggest me different approaches to analyze and different point of view to interpret the results I obtained.