\section{Introduction}

\subsection{Learning from Teacher}

% Imagine you are in a foreign country and you want to take a beverage from a vending machine using an alphabet you don't know. The first thing you do is trying to make some attemps to find the right procedure to select the product you want. At the end thanks to the answers given by the vending machine you will be able to understand the right path of actions to follow in order to get your beverage.

One of the goals of computer science is to create clever algorithms able to adapt theirshelves to not known environment. We can say that humans have the natural capacity to increase their knowledge thanks to intuition and memory and education is a good example of how information is passed from teachers to students.

It is interesting to simulate this precept and recreate the \textit{education} system thanks to two different entities, a \learner{} willing to learn an unknown language \textit{L} and a \teacher{}, the owner of \textit{L}.
% create a system able to understand an unknown language performing actions, registering their consequence until it has enough information to say that the language is now learnt. This process can be represented by a \learner{}, the system willing to learn the new language, and a \teacher{}, the oracle containing the unknwon language which will answer about the validity or not of the received question.

The interactions between \learner{} and \teacher{} are various, for instance, \teacher{} may give informations to the \learner{} without being solliceted, or can provide not relible or useless informations and so on and mutually the \learner{} may not make well-formed questions or may not pay attention to the received answers.

In this project, we are going to reduce the uncountable ways of communication between these two entities and allow only two kind of interactions : membership queries where the \learner{} asks if a sentence belongs to the unknown language and the \teacher{} can answer \textit{yes} or \textit{no} and
equivalence queries where the \learner{} sends to the teacher the language lerant (in the form of an automaton) and the \teacher{} answers either \textit{yes} if if recognizes the good language, otherwise provides a counter-exmple.

\subsection{Structure of this report}
In this report we are going to illustrate briefly the concept of \textit{Finate State Automata} defining "regular languages".

Then we are going to introduce the L* \cite{angluinL} and the NL* \cite{angluinNL} algorithms and finally we compare them in term of performances and quality of results.